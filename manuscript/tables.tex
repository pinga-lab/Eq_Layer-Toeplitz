\begin{table}[h]
	\label{tab:RAM-usage}
	\begin{center}
		\begin{tabular}{|l|c|c|c|}
			\hline
			\textbf{$N$} & \textbf{Full matrix $\mathbf{A}$} & \textbf{Single column of $\mathbf{A}$} 
			& \textbf{Full matrix $\mathbf{L}$}\\
			\hline 
			$100$ & 0.0763 & 0.0000763 & 0.0006104\\
			\hline
			$400$ & 1.22 & 0.0031 & 0.0248\\
			\hline
			$2\,500$ & 48 & 0.0191 & 0.1528\\
			\hline
			$10\,000$ & 763 & 0.00763 & 0.6104\\
			\hline
			$40\,000$ & 12\,207 & 0.305 & 2.4416 \\
			\hline
			$250\,000$ & 476\,837 & 1.907 & 15.3 \\
			\hline
			$500\,000$ & 1\,907\,349 & 3.815 & 30.518 \\
			\hline
			$1\,000\,000$ & 7\,629\,395 & 7.629 & 61.035 \\
			\hline
		\end{tabular}
		\caption{Comparison between the system RAM memory usage needed to store the full matrix $\mathbf{A}$ 
		(equations \ref{eq:predicted-data-vector} and \ref{eq:BTTB_A}), a single column of $\mathbf{A}$ and the 
		full matrix $\mathbf{L}$ (equation \ref{eq:DTF_G}). The quantities are represented in megabyte (MB), for 
		different total number of observations $N$. We consider that the elements of $\mathbf{A}$ and $\mathbf{L}$ 
		occupy, respectively, 8 and 16 bytes each in computer memory.}
	\end{center}
\end{table} 
