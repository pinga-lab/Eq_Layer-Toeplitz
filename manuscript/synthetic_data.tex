\section{Synthetic tests}


In this section, we  investigate the effectiveness of using the properties of BTTB and BCCB matrices 
(equation \ref{eq:w_Cv})  to solve, at each iteration, the forward modeling (the matrix-vector product 
$\mathbf{A} \hat{\mathbf{p}}^k$)  required in the fast equivalent-layer method proposed by \citet{siqueira-etal2017}. 
We simulated three sources whose horizontal projections are shown in Figure \ref{fig:synthetic_data} as black lines. 
These sources are two vertical prisms with density contrasts of $0.35\, \mathrm{g/cm^3}$ (upper-left prism) and 
$0.4\, \mathrm{g/cm^3}$ (upper-right prism) and a sphere with radius of $1\,000$ m with density contrast of 
$-0.5\, \mathrm{g/cm^3}$. Figure \ref{fig:synthetic_data}  shows the vertical component of gravity field generated 
by these sources contaminated with additive pseudorandom Gaussian noise with zero mean and standard deviation of 
$0.015$ mGal.
 
The advantage of using the structures of BTTB and BCCB matrices to compute forward modeling in  the fast-equivalent 
layer method \cite[]{siqueira-etal2017} is grounded on the use of regular grids of data and equivalent sources. 
Hence, we created $10\,000$ observation points regularly spaced in a grid of $100 \times 100$ at $100$ m height. 
We also set a grid of equivalent point masses, each one directly beneath each observation points, located at 
$300$ m deep.  Figures \ref{fig:classic_fast_val}a and \ref{fig:bccb_fast_val}a  show the fitted gravity data 
obtained, respectively, by the fast equivalent layer method and by our modified form of this method that computes 
the forward modeling using equation \ref{eq:w_Cv}. The corresponding residuals (Figures \ref{fig:classic_fast_val}b 
and \ref{fig:bccb_fast_val}b), defined as the difference between the observed (Figure  \ref{fig:synthetic_data}) 
and fitted gravity data (Figures \ref{fig:classic_fast_val}a and \ref{fig:bccb_fast_val}a), show means close to 
zero  and standard deviations of 0.0144 mGal.  Therefore, Figures \ref{fig:classic_fast_val} and 
\ref{fig:bccb_fast_val} show that \citet{siqueira-etal2017} method and our modified version of this method produced 
virtually the same results. This excellent agreement is confirmed in Figures \ref{fig:delta_comparison} and 
\ref{fig:delta_rho} which shows that there are virtually no differences, respectively,  in the fitted data presented 
in Figures \ref{fig:classic_fast_val}b and \ref{fig:bccb_fast_val}b and in the estimated mass distributions within 
the equivalent layers (not shown) yielded by both \citet{siqueira-etal2017} method and our modification of this method. 
These results (Figures \ref{fig:delta_comparison} and \ref{fig:delta_rho})  show that computing the matrix-vector 
product ($\mathbf{A} \hat{\mathbf{p}}^k$), required in the forward modeling, by means of embedding the BTTB matrix into 
a BCCB matrix (equation \ref{eq:w_Cv}) yields practically the same result as the one produced by computing this 
matrix-vector product  with a full matrix $\mathbf{A}$ as used in \citet{siqueira-etal2017}.

We perform two forms of processing the gravity data (Figure \ref{fig:synthetic_data}) through the equivalent layer 
technique: the upward (Figure \ref{fig:upward_comparison}) and the downward (Figure \ref{fig:downward_comparison}) 
continuations. The upward height is 300 m and the downward  is at 50 m.   Either in the upward continuation 
(Figure \ref{fig:upward_comparison}) or in the downward continuation (Figure \ref{fig:downward_comparison}), the 
continued gravity data using the fast equivalent layer proposed by \citet{siqueira-etal2017} 
(Figures \ref{fig:upward_comparison}a and \ref{fig:downward_comparison}a) are in close agreement with those produced 
by our modification of \citet{siqueira-etal2017} method (Figures \ref{fig:upward_comparison}b and 
\ref{fig:downward_comparison}b). The residuals (Figures \ref{fig:upward_comparison}c and \ref{fig:downward_comparison}c)
quantify this agreement since their means and standard deviations are close to zero in both continued gravity data using 
both methods.  All the continued gravity data shown here (Figures \ref{fig:upward_comparison} and 
\ref{fig:downward_comparison}) agree with the true ones (not shown). The most striking feature of these upward or 
the downward continuations concerns the total computation time. The computation time spent by our method is 
approximately $1\,500$ times faster than \citet{siqueira-etal2017} method. 