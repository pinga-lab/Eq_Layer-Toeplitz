\section{Application to synthetic data}

We simulated three sources whose horizontal projections are shown in Figure \ref{fig:synthetic_data} as black lines. 
These sources are a sphere with radius of $1\,000$ m with density contrast of 
$-0.5\, \mathrm{g/cm^3}$ and two rectangular prisms with density contrasts of $0.35\, \mathrm{g/cm^3}$ (upper-left prism) 
and $0.4\, \mathrm{g/cm^3}$ (upper-right prism). 
Figure \ref{fig:synthetic_data} shows the gravity disturbance (vertical component of gravitational attraction) 
produced by these sources. The synthetic data are contaminated with additive pseudorandom Gaussian 
noise with zero mean and standard deviation of $0.015$ mGal.
We created $N = 10\,000$ observation points regularly spaced on a grid of $100 \times 100$, on a 
horizontal plane at $z_{1} = -100$ m, and set a grid of equivalent point masses, each one directly beneath each 
observation point, located at depth $z_{0} = 300$ m. 

Figures \ref{fig:classic_fast_val}a and \ref{fig:bccb_fast_val}a show the data fits 
obtained, respectively, by the fast equivalent-layer technique \citep{siqueira-etal2017} 
and by our method. The corresponding residuals (Figures \ref{fig:classic_fast_val}b 
and \ref{fig:bccb_fast_val}b), defined as the difference between the simulated (Figure \ref{fig:synthetic_data}) 
and predicted data (Figures \ref{fig:classic_fast_val}a and \ref{fig:bccb_fast_val}a) after $N^{it} = 23$ iterations, 
show means close to zero and standard deviations of $0.0144$ mGal, indicating that both methods produced virtually 
the same results.
This excellent agreement is confirmed by Figures \ref{fig:delta_comparison} and 
\ref{fig:delta_rho}. They show that there are virtually no differences between, respectively, 
the predicted data (Figures \ref{fig:classic_fast_val}b and \ref{fig:bccb_fast_val}b) and the 
estimated mass distributions within the equivalent layers (not shown) yielded by 
both methods. 

We performed the upward and downward continuations of the simulated gravity data 
(Figure \ref{fig:synthetic_data}) by using the estimated equivalent layers.
The results show that the continued gravity data obtained by using the 
fast equivalent-layer (Figures \ref{fig:upward_comparison}a and \ref{fig:downward_comparison}a) 
are in close agreement with those produced by our method (Figures \ref{fig:upward_comparison}b and 
\ref{fig:downward_comparison}b). The residuals (Figures \ref{fig:upward_comparison}c and \ref{fig:downward_comparison}c)
quantify this agreement since their means and standard deviations are close to zero. 

The continued gravity data obtained with both methods agree with the true ones (not shown).
The difference is that the computation time spent by our method is 
approximately $1\,500$ times smaller than that of the fast-equivalent-layer 
technique \citep{siqueira-etal2017}. 