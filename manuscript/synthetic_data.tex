\section{Application to synthetic data}

We simulated three sources whose horizontal projections are shown in Figure \ref{fig:synthetic_data} as black lines. 
These sources are a sphere with radius of $1\,000$ m with density contrast of 
$-0.5\, \mathrm{g/cm^3}$ and two rectangular prisms with density contrasts of $0.35\, \mathrm{g/cm^3}$ (upper-left prism) 
and $0.4\, \mathrm{g/cm^3}$ (upper-right prism). 
Figure \ref{fig:synthetic_data} shows the gravity disturbance (vertical component of gravitational attraction) 
produced by these sources. The synthetic data are contaminated with additive pseudorandom Gaussian 
noise with zero mean and standard deviation of $0.015$ mGal.
We created $N = 10\,000$ observation points regularly spaced on a grid of $100 \times 100$, on a 
horizontal plane at $z_{1} = -100$ m, and set a grid of equivalent point masses, each one directly beneath each 
observation point, located at depth $z_{0} = 300$ m. 

Figures \ref{fig:comparison_datafits}a and \ref{fig:comparison_datafits}b show the data fits 
obtained, respectively, by the fast equivalent-layer technique \citep{siqueira-etal2017} 
and by our method. They represent the differences between the simulated data (Figure \ref{fig:synthetic_data}) and the predicted data produced by both methods 
(not shown) after $N^{it} = 23$ iterations. 
As we can see, both methods produced virtually the same results.
This excellent agreement is confirmed by Figure \ref{fig:comparison_datafits}c, which shows 
the differences between the predicted data produced by both methods.

We performed the upward and downward continuations of the simulated gravity data 
(Figure \ref{fig:synthetic_data}) by using the fast equivalent-layer technique 
and our method.
Figure \ref{fig:upward_comparison} shows that the upward-continued gravity data obtained 
by using both methods are in close agreement with those produced by the synthetic 
bodies at $z = XXXX$ m (Figure \ref{fig:upward_comparison}a).
As we can see in Figures \ref{fig:upward_comparison}b and \ref{fig:upward_comparison}c,
the residuals between the upward-continued and true data have their means and standard 
deviations close to zero. 
Figure \ref{fig:downward_comparison} shows similar results obtained by using both 
methods to compute downward continuation of gravity data.
It is important to stress that, although both methods are able to produce a reliable 
upward/downward continuation of gravity data, the computation time spent by our method 
is approximately $1\,500$ times smaller than that of the fast equivalent-layer 
technique \citep{siqueira-etal2017}. 