\section{Application to synthetic data}

We have simulated three sources whose horizontal projections are shown in 
Figure \ref{fig:synthetic_data} as black lines. 
These sources are a sphere with density contrast  
$-1.25\, \mathrm{g/cm^3}$ and two rectangular prisms with density contrasts 
$1.00\, \mathrm{g/cm^3}$ (upper-left prism) 
and $1.30\, \mathrm{g/cm^3}$ (upper-right prism). 
Figure \ref{fig:synthetic_data} shows the gravity disturbance (vertical component of 
gravitational attraction) produced by these sources. 
The synthetic data are contaminated with additive pseudorandom Gaussian 
noise with zero mean and standard deviation of $0.1 \, \mathrm{mGal}$.
The data are computed at $N = 10\,000$ observation points that are regularly spaced on a 
$100 \times 100$ grid, at $z_{1} = -100$ m. 
We have set a grid of equivalent sources, each one directly beneath each 
observation point, at $z_{0} = 300$ m. 

Figures \ref{fig:comparison_datafits}a and \ref{fig:comparison_datafits}b show the data fits 
obtained, respectively, by the fast equivalent-layer technique \citep{siqueira-etal2017} 
and by our method. They represent the differences between the simulated data (Figure \ref{fig:synthetic_data}) and the predicted data produced by both methods 
(not shown) after $N^{it} = 40$ iterations. 
As we can see, both methods produce virtually the same results.
This excellent agreement is confirmed by Figure \ref{fig:comparison_datafits}c, which shows 
the differences between the predicted data produced by both methods.

We performed the upward and downward continuations of the simulated gravity data 
(Figure \ref{fig:synthetic_data}) by using the fast equivalent-layer technique
\citep{siqueira-etal2017}, our method and also the classical approach in the 
Fourier domain. This approach consists in performing the upward/downward continuation 
by directly computing the Fourier transform of the gravity data 
\citep[e.g., ][ p. 317]{blakely1996}.
Figure \ref{fig:upward_comparison} shows the upward-continued gravity data 
obtained by the three methods.
As we can see, the residuals between the true data at $z = -300$ m 
(Figure \ref{fig:upward_comparison}a) and the upward-continued data obtained by 
using our method (Figure \ref{fig:upward_comparison}b) and 
the fast equivalent-layer technique (Figure \ref{fig:upward_comparison}c) are very similar 
to each other.
Notice that the absolute values of the residuals produced by the classical Fourier 
approach (Figure \ref{fig:upward_comparison}d) are $\approx 10$ times greater than
those produced by our method and the fast equivalent-layer technique 
(Figures \ref{fig:upward_comparison}b and \ref{fig:upward_comparison}c), 
with maximum values concentrated at the border of the simulated area.
We opted for showing the results produced by all methods without removing the border 
effects in order to properly compare them to each other.
We can also notice that, differently from the results produced by the other methods, 
that obtained with the classical Fourier approach exhibits a slight noise amplification.

Figure \ref{fig:downward_comparison} shows the results obtained by using all 
methods to compute downward continuation of gravity data.
In this case, the maximum absolute values of the residuals produced by the 
classical Fourier approach (Figure \ref{fig:downward_comparison}d) are 
$\approx 20$ times greater than those produced by our method and the fast equivalent-layer technique (Figures \ref{fig:downward_comparison}b and \ref{fig:downward_comparison}c).
It is also important to stress that the result produced by the classical Fourier approach 
exhibits a considerable noise amplification when compared to the other methods.
This noise amplification is a well-known problem of the downward continuation 
produced by the classical Fourier approach \citep[e.g., ][ p. 320]{blakely1996}.

Another important aspect to be pointed out about the results we obtained with 
synthetic data (Figures \ref{fig:upward_comparison} and \ref{fig:downward_comparison})
is the computational times spent by our method and the fast equivalent-layer technique. 
The total computational time required by our method to estimate the physical property 
distribution on the equivalent layer and perform the upward/downward continuation 
is about two orders of magnitude lower than that spent by the fast equivalent-layer technique.
This significant reduction in computational time was obtained by using a data set 
composed of $N = 10\,000$ observation points. Considerably better results can be 
obtained with larger data sets.

Finally, we did not compare the total computational times spent by our method and the classical 
Fourier approach, but we can affirm that the second is faster. 
While the classical approach needs basically one DFT/IDFT of the data, 
our method requires one DFT/IDFT per iteration. 
On the other hand, the considerably smaller noise amplification and 
border effect are the main advantages of our method over the classical Fourier 
approach, especially in the downward continuation.

