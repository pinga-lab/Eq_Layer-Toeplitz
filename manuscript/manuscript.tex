%\documentclass[paper,twocolumn,twoside]{geophysics}
\documentclass[manuscript,blind]{geophysics}
%\documentclass[manuscript]{geophysics}

% An example of defining macros
\newcommand{\rs}[1]{\mathstrut\mbox{\scriptsize\rm #1}}
\newcommand{\rr}[1]{\mbox{\rm #1}}

% Extra packages
\usepackage{amsmath}
\usepackage{bm}
\usepackage[pdftex,colorlinks=true]{hyperref}
\hypersetup{
	allcolors=black,
}
\usepackage{lipsum}
\usepackage[table]{xcolor}

\begin{document}

\title{Convolutional equivalent layer for gravity data processing}

\renewcommand{\thefootnote}{\fnsymbol{footnote}} 

\ms{GEO-XXXX} % manuscript number

\address{
\footnotemark[2]Observat\'{o}rio Nacional, Department of Geophysics, Rio de Janeiro, Brazil\\
\footnotemark[1] Corresponding author: diego.takahashi@gmail.com
}
\author{Diego Takahashi\footnotemark[2]\footnotemark[1], Vanderlei C. Oliveira{ }Jr.\footnotemark[2] and 
Val{\'e}ria C. F. Barbosa\footnotemark[2]}

%\footer{Example}
%\lefthead{Takahashi, Oliveira{ }Jr. \& Barbosa}
\righthead{Convolutional equivalent layer}

\maketitle

% Main body
\begin{abstract}

We have developed an efficient and very fast equivalent-layer technique for gravity data processing by modifying an 
iterative method grounded on excess mass constraint that does not require the solution of linear systems. 
Taking advantage of the symmetric Block-Toeplitz Toeplitz-block (BTTB) structure of the sensitivity matrix, that raises 
when regular grids of observation points and equivalent sources (point masses) are used to set up a fictitious 
equivalent layer, we have developed an algorithm which greatly reduces the number of flops and RAM memory necessary 
to estimate a 2D mass distribution over the equivalent layer. The structure of symmetric BTTB matrix consists of 
the elements of the first column of the sensitivity matrix, which in turn can be embedded into a symmetric 
Block-Circulant Circulant-Block (BCCB) matrix. Likewise, only the first column of the BCCB matrix is needed 
to reconstruct the full sensitivity matrix completely. From the first column of BCCB matrix, its eigenvalues 
can be calculated using the 2D Fast Fourier Transform (2D FFT), which can be used to readily 
compute the matrix-vector product of the forward modeling in the fast equivalent-layer technique. As a result, 
our method is efficient to process very large datasets using either fine- or mid-grid meshes. The larger the 
dataset, the faster and more efficient our method becomes compared to the available equivalent-layer techniques. 
Tests with synthetic data demonstrate the ability of our method to satisfactorily upward-
and downward-continuing the gravity data.
Our results show very small border effects and noise amplification compared to those 
produced by the classical approach in the Fourier domain.
Besides, they show that while the running time of our method is $\approx 30.9$ seconds for 
processing $N = 1\,000\,000$ observations, the fast equivalent-layer technique spent 
$\approx 46.8$ seconds with $N = 22\,500$.
A test with field data from Caraj{\'a}s province, Brazil, illustrates the low computational 
cost of our method to process a large data set composed of $N = 250\,000$ observations.

\end{abstract}
\section{Introduction}

The equivalent layer is a well-known technique for processing potential-field data in applied geophysics 
since the 60's. 
It comes from potential theory as a mathematical solution of the Laplace's equation, in the region above the
sources, by using the Dirichlet boundary condition \citep{kellogg1929}.
This theory states that any potential-field data produced by an arbitrary 3D physical-property distribution can 
be exactly reproduced by a fictitious layer located at any depth and having a continuous 2D physical-property  
distribution. In practical situations, the layer is approximated by a finite set of sources (e.g., point masses 
or dipoles) and their physical properties are estimated by solving a linear system of equations that yields an 
acceptable potential-field data fit. These fictitious sources are called equivalent sources.

Many previous works have used the equivalent layer to perform different potential-field data 
transformations such as gridding \citep[e.g.,][]{dampney1969, cordell1992, mendonca-silva1994},
upward/downward continuation \citep[e.g.,][]{emilia1973, hansen-miyazaki1984, 
li-oldenburg2010}, reduction to the pole \citep[e.g.,]{silva1986, leao-silva1989, guspi-novara2009, 
oliveirajr-etal2013}, combining multiple data sets \citep[e.g.,][]{boggs-dransfield2004} and 
gradient data processing \citep[e.g.,][]{barnes-lumley2011}.

Although the use of the equivalent-layer technique increased over the last decades, one of the biggest 
problems is still its high computational cost for processing large-data sets. 
This problem propelled several studies to improve the computational efficiency of the 
equivalent layer technique. 
\citet{leao-silva1989} developed a fast method for processing a regular grid of potential-field data.
The method consists in estimating an equivalent layer which exactly reproduces the potential-field data 
within a small data window. The data window is shifted over the whole gridded data set in a procedure similar 
to a discrete convolution. 
The equivalent layer extends beyond the moving-data window and is located at a depth between two and six 
times the grid spacing of the observations. For each data window, the equivalent layer is estimated by 
solving an underdetermined linear system.
After estimating an equivalent layer, the transformed-potential field is computed only at the center of the 
moving-data window. The use of a small moving-data window greatly reduces the total number of floating-point 
operations (\textit{flops}) and RAM memory storage. The computational efficiency of this method relies on the 
strategy of constructing the equivalent layer by successively solving small linear systems instead of solving 
just one large linear system for the entire equivalent layer. 
\citet{mendonca-silva1994} also followed the strategy of solving successive small linear systems for 
constructing an equivalent layer. Their method is based on the  equivalent-data concept, which 
consists in determining a subset of all potential-field data (named equivalent-data set), such that the 
interpolating surface that fits the chosen subset also automatically fits all remaining data. 
The equivalent data set is obtained by iteratively introducing the potential-field observation with the 
greatest residual in the preceding iteration. By applying to the interpolation problem, the method is 
optimized by approximating dot products by the discrete form of an analytic integration that can be 
evaluated with less computational effort. According to the authors, the equivalent-data set is usually 
smaller than the total number of potential-field observations, leading to computational savings. 
The authors also pointed out that the computational efficiency of the method depends on the number of 
equivalent data. If the potential-field anomaly is nonsmooth, the number of equivalent data can be large and 
the method will be less efficient than the classical approach.

By following a different strategy, \citet{li-oldenburg2010} developed a rapid method that transforms the 
dense sensitivity matrix associated with the linear system into a sparse one by using a wavelet technique. 
After obtaining a sparse representation of the sensitivity matrix, those authors estimate the 
physical-property distribution within the equivalent layer by using an overdetermined formulation. 
Those authors pointed out that, given the sparse representation, their method reduces the computational time 
required for solving the linear system by as many as two orders of magnitude if compared with the same 
formulation using a dense matrix. 
\citet{barnes-lumley2011} followed a similar strategy and transformed the dense sensitivity matrix 
into a sparse one. However, differently from \citet{li-oldenburg2010}, their method operates in the space 
domain by grouping equivalent sources far from an observation point into blocks with average physical 
property. This procedure aims at reducing the memory storage and achieving computational efficiency 
by solving the transformed linear system with a weighted-least-squares conjugate-gradient algorithm.
Notice that, instead of constructing the equivalent layer by solving successive small linear systems, 
these last two methods first transform the large linear system into a sparse one and then take advantage of 
this sparseness.

\citet{oliveirajr-etal2013} developed a fast method based on the reparameterization of the physical-property 
distribution within the equivalent layer. Those authors divided the equivalent layer into a regular grid of 
equivalent-source windows inside which the physical-property distribution is described by bivariate 
polynomial functions. By using this polynomial representation, the inverse problem for estimating the 
equivalent layer is posed in the space of the total number of polynomial coefficients within all 
equivalent-source windows instead of in the space of the total number of equivalent sources. According to 
\citet{oliveirajr-etal2013}, the computational efficiency of their method relies on the fact that the total 
number of polynomial coefficients needed to describe the physical property distribution within the 
equivalent layer is generally much smaller than the number of equivalent sources, leading to a very smaller 
linear system to be solved. Those authors could verify that the total number of \textit{flops} needed for 
building and solving the linear inverse problem of estimating the total number of polynomial coefficients can 
be reduced by as many as three and four orders of magnitude, respectively, if compared with the same inverse 
problem of estimating the physical property of each equivalent source via Cholesky decomposition.

There is another class of methods that iteratively estimates the physical-property distribution within 
the equivalent layer without solving linear systems.
The method presented by \citet{cordell1992} and after generalized by \citet{guspi-novara2009} updates 
the physical property of the sources, which are located below each potential-field data, using a 
procedure that removes the maximum residual between the observed and predicted data.
\citet{xia-sprowl1991} and \citet{xia-etal1993} developed 
fast iterative schemes for updating the physical-property distribution
within the equivalent layer in the wavenumber and space domains, respectively.
Grounded on excess mass constraint, \cite{siqueira-etal2017} developed an iterative scheme 
starting with a mass distribution within the equivalent layer that is proportional to observed gravity data.
Then, their method iteratively adds mass corrections that are proportional to the gravity residuals.
The total number of \textit{flops} required by these iterative methods for estimating the physical-property 
distribution within the equivalent layer depends on the total number of iterations, however this number is 
generally much smaller than the total number of \textit{flops} required to solve a large-scaled linear system. 
Generally, the most computational expensive step in each iteration of these methods is the forward problem 
of calculating the potential-field data produced by the equivalent layer.

In the present work, we show that the sensitivity matrix associated with a planar equivalent layer 
of point masses has a very well-defined structure called Block-Toeplitz Toeplitz-Block (BTTB) for 
the case in which (i) the observed gravity data is located on a regularly spaced grid at constant 
height and (ii) there is one point mass directly beneath each observation point.
This technique have been successfully used in potential-field methods for 
3D gravity inversion \citep{zhang-wong2015}, downward continuation 
\citep{zhang-etal2016} and 3D magnetic modeling \citep{qiang_etal2019}.
By using this property, we propose an efficient algorithm based on fast FFT convolution 
\citep[e.g.,][ p. 207]{vanloan1992} for computing the forward problem at each iteration of 
the fast equivalent-layer technique proposed by \citet{siqueira-etal2017}.
Our method uses the gravitational effect produced by a single point mass to compute the 
effect produced by the whole equivalent layer, which results in a drastic reduction 
not only in the number of flops, but also in the RAM memory usage of the fast equivalent-layer technique.
Tests with synthetic and field data illustrate the good performance of our method in processing 
large gravity data sets.




\section{Methodology}

\subsection{Equivalent-layer technique for gravity data processing}

Let $d^{o}_{i}$ be the observed gravity data at
the point $(x_{i}, y_{i}, z_{i})$, $i = 1, ..., N$, of a local Cartesian
system with $x$ axis pointing to north, the $y$ axis pointing to east and 
the $z$ axis pointing downward.
Let us consider an equivalent layer compose by a set of $N$ point masses 
(equivalent sources) over a layer located at depth $z_0$ ($z_0 >z_i$) and whose 
$x$- and $y$- coordinates of each point mass coincide with the corresponding coordinates 
of the observation directly above.
There is a linear relationship that maps the unknown mass distribution onto the gravity 
data given by
\begin{equation}
\mathbf{d}(\mathbf{p}) = \mathbf{A} \mathbf{p} \: ,
\label{eq:predicted-data-vector}
\end{equation}
where $\mathbf{d}$ is an $N \times 1$ vector whose $i$th element is the predicted gravity 
data at the $i$th point ($x_i$,$y_i$,$z_i$), $\mathbf{p}$ is the unknown $N \times 1$ 
parameter vector whose $j$th element $p_j$  is the mass of the $j$th equivalent source 
(point mass) at the $j$th Cartesian coordinates ($x_j$,$y_j$,$z_0$) and $\mathbf{A}$ 
is an $N \times N$  sensitivity matrix whose $ij$th element is given by 
\begin{equation}
a_{ij}= \frac{c_{g} \, G \, (z_{0} - z_{i})}{\left[(x_{i} - x_{j})^{2} +
(y_{i} - y_{j})^{2} +	(z_{i} - z_{0})^{2} \right]^{\frac{3}{2}}} \; ,
\label{eq:aij}
\end{equation}
where $G$ is the Newton's gravitational constant and $c_{g} = 10^{5}$ 
transforms from $\mathrm{m/s^2}$ to mGal.
Notice that the sensitivity matrix depends on the $i$th coordinates of the observations 
and the $j$th coordinates of the equivalent sources. For convenience, we designate 
these coordinates as \textit{matrix coordinates} and the indices $i$ and $j$ as 
\textit{matrix indices}.
In the classical equivalent-layer technique, we estimate the regularized parameter vector 
from the observed gravity data $\mathbf{d}^{o}$ by
\begin{equation}
\hat{\mathbf{p}} = \left( \mathbf{A}^{\top}\mathbf{A} + 
\mu \, \mathbf{I} \right)^{-1}
\mathbf{A}^{\top} \mathbf{d}^{o} \: .
\label{eq:estimated-p-parameter-space}
\end{equation}

\subsection{Fast equivalent-layer technique}
\citet{siqueira-etal2017} developed an iterative least-squares method to estimate the mass 
distribution over the equivalent layer based on the excess of mass and the positive correlation 
between the observed gravity data and the masses on the equivalent layer. Those authors showed 
that the fast equivalent-layer technique has a better computational efficiency than the 
classical equivalent layer approach (equation \ref{eq:estimated-p-parameter-space}) if the 
dataset totalize at least 200 observation points, even using a large number of iterations.

Considering one equivalent source (point mass) directly beneath each observation point, 
the iteration of the \citeauthor{siqueira-etal2017}'s~(\citeyear{siqueira-etal2017}) method 
starts by an initial approximation of mass distribution given by
\begin{equation}
\hat{\mathbf{p}}^0 = \tilde{\mathbf{A}}^{-1} \mathbf{d}^{o} \: ,
\label{eq:p0_fast_eqlayer}
\end{equation}
where $\tilde{\mathbf{A}}^{-1}$ is an $N \times N$ diagonal matrix with elements
\begin{equation}
\tilde{a}_{ii}^{-1} = \frac{\Delta s_i}{(2 \pi \, G \, c_g)} \: ,
\label{eq:aii_tilde_inv_fast_eqlayer}
\end{equation}
where $\Delta s_i$ is the $i$th element of surface area located at the $i$th horizontal 
coordinates $x_i$ and $y_i$ of the $i$th observation.
At the $k$th iteration, the masses of the equivalent sources are updated by
\begin{equation}
\hat{\mathbf{p}}^{k+1} = \hat{\mathbf{p}}^{k} + \mathbf{\Delta} \hat{\mathbf{p}}^{k} \: ,
\label{eq:p_k+1_fast_eqlayer}
\end{equation}
where the mass correction is given by
\begin{equation}
\mathbf{\Delta} \hat{\mathbf{p}}^{k+1} = \tilde{\mathbf{A}}^{-1} (\mathbf{d}^{o} - \mathbf{A} \hat{\mathbf{p}}^{k}) \: .
\label{eq:delta_p_k_fast_eqlayer}
\end{equation}

At the $k$th iteration of \citeauthor{siqueira-etal2017}'s~(\citeyear{siqueira-etal2017}) method, 
the matrix-vector product 
$\tensor{A} \hat{\mathbf{p}}^{k} = \mathbf{d}(\hat{\mathbf{p}}^{k})$ must be calculated to get a 
new residual $\mathbf{d^0} - \tensor{A} \hat{\mathbf{p}}^{k}$, which represents a bottleneck. 
Considering the limitation of 16 Gb of RAM 
memory in our system, we could run their method only up to 22 500 observation points. 
Hence, for very large data sets it is costful and can be overwhelming in terms of RAM 
memory to maintain such operation.

\subsection{Structure of matrix $\mathbf{A}$ for regular grids}

Consider that the observed data are located on an $N_{x} \times N_{y}$ regular grid of
points regularly spaced from $\Delta x$ and $\Delta y$ along the $x$ and $y$ directions,
respectively, on a horizontal plane defined by the constant vertical coordinate $z_{1} < z_{0}$. 
As a consequence, a given pair of matrix coordinates $(x_{i}, y_{i})$, defined by the matrix index 
$i$, $i = 1, \dots, N = N_{x} N_{y}$, is equivalent to a pair of coordinates $(x_{k}, y_{l})$
given by:
\begin{equation}
x_{i} \equiv x_{k} = x_{1} + \left[ k(i) - 1 \right] \, \Delta x \: , 
\label{eq:xi}
\end{equation}
and
\begin{equation}
y_{i} \equiv y_{l} = y_{1} + \left[ l(i) - 1 \right] \, \Delta y \: ,
\label{eq:yi}
\end{equation}
where $k(i)$ and $l(i)$ are integer functions of the matrix index $i$.
These equations can also be used to define the matrix coordinates 
$x_{j}$ and $y_{j}$ associated with the $j$-th equivalent source,
$j = 1, \dots, N = N_{x}N_{y}$. In this case, the integer functions
are evaluated by using the index $j$ instead of $i$.
For convenience, we designate $x_{k}$ and $y_{l}$ as \textit{grid coordinates}
and the indices $k$ and $l$ as \textit{grid indices}, which are computed with
the integer functions.

The integer functions assume different forms depending on the 
orientation of the regular grid of data.
Consider the case in which the grid is oriented along the
$x$-axis (Figure 1a). For convenience, we designate these grids as 
$x$-\textit{oriented grids}. For them, we have the following integer functions:
\begin{equation}
i(k, l) = (l - 1) \, N_{x} + k \quad ,
\label{eq:i-x-oriented}
\end{equation}
\begin{equation}
l(i) = \Bigg\lceil \frac{i}{N_{x}} \Bigg\rceil
\label{eq:l-x-oriented}
\end{equation}
and
\begin{equation}
k(i)  = i - \Bigg\lceil \frac{i}{N_{x}} \Bigg\rceil N_{x} + N_{x} \quad ,
\label{eq:k-x-oriented}
\end{equation}
where $\lceil \cdot \rceil$ denotes the ceiling function \citep[][ p. 67]{graham-etal1994}.
These integer functions are defined in terms of the matrix index $i$, but they can 
be defined in the same way by using the index $j$.
Figure 1a illustrates an $x$-oriented grid defined by $N_{x} = 4$ and $N_{y} = 3$.
In this example, the matrix coordinates $x_{7}$ and $y_{7}$, defined by the matrix index $i = 7$ (or $j = 7$), 
are equivalent to the grid coordinates $x_{3}$ and $y_{2}$, which are defined by the grid indices
$k = 3$ and $l = 2$, respectively. These indices are computed with equations \ref{eq:l-x-oriented}
and \ref{eq:k-x-oriented}, by using the matrix index $i = 7$ (or $j = 7$).

Now, consider the case in which the regular grid of data is oriented along 
the $y$-axis (Figure 1b). For convenience, we call them $y$-\textit{oriented grids}.
Similarly to $x$-oriented grids, we have the following integer functions associated with
$y$-oriented grids:
\begin{equation}
i(k, l) = (k - 1) \, N_{y} + l \quad ,
\label{eq:i-y-oriented}
\end{equation}
\begin{equation}
k(i) = \Bigg\lceil \frac{i}{N_{y}} \Bigg\rceil
\label{eq:k-y-oriented}
\end{equation}
and
\begin{equation}
l(i) = i - \Bigg\lceil \frac{i}{N_{y}} \Bigg\rceil N_{y} + N_{y} \quad .
\label{eq:l-y-oriented}
\end{equation}
Figure 1b illustrates an $y$-oriented grid defined by $N_{x} = 4$ and $N_{y} = 3$.
In this example, the matrix coordinates $x_{7}$ and $y_{7}$, defined by the matrix index 
$i = 7$ (or $j = 7$), are equivalent to the grid coordinates $x_{3}$ and $y_{1}$, which are 
defined by the grid indices $k = 3$ and $l = 1$, respectively. Differently from the example
shown in Figure 1a, the grid indices of the present example are 
computed with equations \ref{eq:k-y-oriented} and \ref{eq:l-y-oriented}, by using the 
matrix index $i = 7$ (or $j = 7$).

The element $a_{ij}$ (equation \ref{eq:aij}) can be rewritten 
by using equations \ref{eq:xi} and \ref{eq:yi}, giving rise to:
\begin{equation}
a_{ij} = \frac{c_{g} \, G \, \Delta z}{ \left[ 
	\left( \Delta k_{ij} \, \Delta x \right)^{2} + 
	\left( \Delta l_{ij} \, \Delta y \right)^{2} + 
	\left( \Delta z \right)^{2} \right]^{\frac{3}{2}}} \: ,
\label{eq:aij-regular-grids}
\end{equation}
where $\Delta z = z_{0} - z_{1}$, 
$\Delta k_{ij} = k(i) - k(j)$ (equations \ref{eq:k-x-oriented} and \ref{eq:k-y-oriented}) and
$\Delta l_{ij} = l(i) - l(j)$ (equations \ref{eq:l-x-oriented} and \ref{eq:l-y-oriented}).
Notice that the structure of matrix $\mathbf{A}$ (equation \ref{eq:predicted-data-vector}) for 
the case in which its elements are given by $a_{ij}$ (equation \ref{eq:aij-regular-grids}) is 
defined by the coefficients $\Delta k_{ij}$ and $\Delta l_{ij}$.

For $x$-oriented grids, the coefficients $\Delta k_{ij}$ and $\Delta l_{ij}$ are 
computed by using equations \ref{eq:k-x-oriented} and \ref{eq:l-x-oriented}, respectively.
In this case, $\mathbf{A}$ (equation \ref{eq:predicted-data-vector}) is 
composed of $N_{y} \times N_{y}$ blocks, where each block is formed by $N_{x} \times N_{x}$ elements.
For $y$-oriented grids, the coefficients $\Delta k_{ij}$ and $\Delta l_{ij}$ are 
computed by using equations \ref{eq:k-y-oriented} and \ref{eq:l-y-oriented}, respectively.
In this case, $\mathbf{A}$ (equation \ref{eq:predicted-data-vector}) is a 
composed of $N_{x} \times N_{x}$ blocks, where each block is formed by $N_{y} \times N_{y}$ elements.
In both cases, $\mathbf{A}$ is Toeplitz blockwise, i.e., the blocks lying at the same block 
diagonal are equal to each other.
Besides, the blocks located above the main diagonal are equal to those 
located below and each block is itself a Toeplitz matrix.
These symmetries come from the fact that the coefficients
$\Delta k_{ij}$ and $\Delta l_{ij}$ are squared at the denominator of 
$a_{ij}$ (equation \ref{eq:aij-regular-grids}).
Matrices with this well-defined pattern are called 
Doubly Block Toeplitz \citep[][ p. 28]{jain1989} or symmetric Block-Toeplitz Toeplitz-Block (BTTB),
for example. We opted for using the second term.

This well-defined pattern is better represented by using the \textit{block indices} $q$ and $p$. 
We represent $\mathbf{A}$ (equation \ref{eq:predicted-data-vector}) as a grid of $Q \times Q$ blocks 
$\mathbf{A}_{q}$, $q = 0, \dots, Q - 1$, given by
\begin{equation}
	\mathbf{A} = \begin{bmatrix}
	\mathbf{A}_{0}   & \mathbf{A}_{1} & \cdots         & \mathbf{A}_{Q-1} \\
	\mathbf{A}_{1}   & \mathbf{A}_{0} & \ddots         & \vdots           \\ 
	\vdots           & \ddots         & \ddots         & \mathbf{A}_{1}   \\
	\mathbf{A}_{Q-1} & \cdots         & \mathbf{A}_{1} & \mathbf{A}_{0}                 
	\end{bmatrix}_{N \times N} \: ,
	\label{eq:BTTB_A}
\end{equation}
where each block has $P \times P$ elements conveniently represented by $a^{q}_{p}$, 
$p = 0, \dots, P - 1$, as follows:
\begin{equation}
	\mathbf{A}_{q} = \begin{bmatrix}
	a^{q}_{0}   & a^{q}_{1} & \cdots    & a^{q}_{P-1} \\
	a^{q}_{1}   & a^{q}_{0} & \ddots    & \vdots           \\ 
	\vdots      & \ddots    & \ddots    & a^{q}_{1}   \\
	a^{q}_{P-1} & \cdots    & a^{q}_{1} & a^{q}_{0}                 
	\end{bmatrix}_{P \times P} \: ,
	\label{eq:Aq_block}
\end{equation}
with $N = QP$. The index $q$ defines the block diagonal where $\mathbf{A}_{q}$ (equation \ref{eq:Aq_block}) 
lies within the BTTB matrix $\mathbf{A}$ (equation \ref{eq:BTTB_A}). 
This index varies from $0$, at the main diagonal, to $Q - 1$, at
the corners of $\mathbf{A}$. Similarly, the index $p$ defines the diagonal where $a^{q}_{p}$ 
lies within $\mathbf{A}_{q}$ (equation \ref{eq:Aq_block}). This index varies from $0$, 
at the main diagonal, to $P - 1$, at the corners of $\mathbf{A}_{q}$.
For $x$-oriented grids, $Q = N_{y}$, $P = N_{x}$ and the block indices
$q$ and $p$ are defined, respectively, by the following integer functions 
of the matrix indices $i$ and $j$:
\begin{equation}
q(i, j) = \; \mid l(i) - l(j) \mid
\label{eq:q-x-oriented}
\end{equation}
and
\begin{equation}
p(i, j) = \; \mid k(i) - k(j) \mid \quad ,
\label{eq:p-x-oriented}
\end{equation}
where $l(i)$ and $l(j)$ are defined by equation \ref{eq:l-x-oriented} 
and $k(i)$ and $k(j)$ are defined by equation \ref{eq:k-x-oriented}.
For $y$-oriented grids, $Q = N_{x}$, $P = N_{y}$ and the block indices
$q$ and $p$ are defined, respectively, by the following integer functions 
of the matrix indices $i$ and $j$:
\begin{equation}
q(i, j) = \; \mid k(i) - k(j) \mid 
\label{eq:q-y-oriented}
\end{equation}
and
\begin{equation}
p(i, j) = \; \mid l(i) - l(j) \mid \quad ,
\label{eq:p-y-oriented}
\end{equation}
where $k(i)$ and $k(j)$ are defined by equation \ref{eq:k-y-oriented}
and $l(i)$ and $l(j)$ are defined by equation \ref{eq:l-y-oriented}.
Notice that, for each element $a_{ij}$ (equation \ref{eq:aij-regular-grids}),
defined by matrix indices $i$ and $j$, there is a corresponding block element
$a^{q}_{p}$, defined by block indices $q$ (equations \ref{eq:q-x-oriented} and 
\ref{eq:q-y-oriented}) and $p$ (equations \ref{eq:p-x-oriented} and \ref{eq:p-y-oriented}),
so that
\begin{equation}
	a^{q}_{p} \equiv a_{ij} \quad .
	\label{eq:aqp_equiv_aij}
\end{equation}
We also stress that matrix $\mathbf{A}$ (equation \ref{eq:predicted-data-vector}) defined 
by elements $a_{ij}$ (equation \ref{eq:aij-regular-grids}) in terms of matrix indices 
$i$ and $j$ is strictly the same BTTB matrix $\mathbf{A}$ (equation \ref{eq:BTTB_A}) defined 
by the blocks $\mathbf{A}_{q}$ (equation \ref{eq:Aq_block}) and block elements 
$a^{q}_{p}$ (equation \ref{eq:aqp_equiv_aij}) in terms of the block indices 
$q$ (equations \ref{eq:q-x-oriented} and \ref{eq:q-y-oriented}) and $p$ 
(equations \ref{eq:p-x-oriented} and \ref{eq:p-y-oriented}).

It is important noting that different matrix indices $i$ or $j$ produce the same 
absolute values for the grid indices $k$ (equations \ref{eq:k-x-oriented} and
\ref{eq:k-y-oriented}) and $l$ (equations \ref{eq:l-x-oriented} and
\ref{eq:l-y-oriented}). As a consequence, different pairs of matrix indices $i$
and $j$ generate the same absolute values for the coefficients $\Delta k_{ij}$ and
$\Delta l_{ij}$ that compose the denominator of $a_{ij}$ 
(equation \ref{eq:aij-regular-grids}) and also the same values for the block indices
$q$ (equations \ref{eq:q-x-oriented} and \ref{eq:q-y-oriented}) and 
$p$ (equations \ref{eq:p-x-oriented} and \ref{eq:p-y-oriented}), as well as the same
block element $a^{q}_{p}$ (equation \ref{eq:aqp_equiv_aij}). 
It means that elements $a_{ij}$ defined
by different matrix indices $i$ and $j$ have the same value. The key point for
understanding the structure of BTTB matrix $\mathbf{A}$ (equation \ref{eq:BTTB_A})
is then, given a single element $a_{ij}$
defined by matrix indices $i$ and $j$, compute the grid indices 
$k$ (equations \ref{eq:k-x-oriented} and \ref{eq:k-y-oriented}) and
$l$ (equations \ref{eq:l-x-oriented} and \ref{eq:l-y-oriented}).
These grid indices are used to 
(1) compute the coefficients $\Delta k_{ij}$ and 
$\Delta l_{ij}$ and determine the value of $a_{ij}$ with equation 
\ref{eq:aij-regular-grids} and 
(2) compute the block indices $q$ (equations \ref{eq:q-x-oriented}, 
and \ref{eq:q-y-oriented}) and $p$ (equations \ref{eq:p-x-oriented} and
and \ref{eq:p-y-oriented}) and determine the corresponding block element $a^{q}_{p}$ 
(equation \ref{eq:aqp_equiv_aij}) forming $\mathbf{A}_{q}$ (equation \ref{eq:Aq_block}). 

Consider the $x$-oriented grid of $N_{x} \times N_{y}$ points shown in Figure 1a, 
with $N_{x} = 4$, $N_{y} = 3$ and $N = N_{x} \, N_{y} = 12$.
To illustrate the relationship between the matrix indices ($i$ and $j$) and 
the block indices ($q$ and $p$), consider the element $a_{ij}$ defined by 
$i = 2$ and $j = 10$, which is
located at the upper right corner of $\mathbf{A}$ (equation \ref{eq:predicted-data-vector}), 
in the 2nd line and 10th column.
By using equations \ref{eq:l-x-oriented} and \ref{eq:k-x-oriented}, we obtain the 
grid indices $l(i) = 1$, $l(j) = 3$, $k(i) = 2$ and $k(j) = 2$.
These grid indices result in the coefficients $\Delta k_{ij} = 0$ and $\Delta l_{ij} = -2$,
which are used to compute the element $a_{ij}$ (equation \ref{eq:aij-regular-grids}),
as well as in the block indices $q = 2$ (equation \ref{eq:q-x-oriented}) and 
$p = 0$ (equation \ref{eq:p-x-oriented}).
These block indices indicate that this element $a_{ij}$ appears in the main diagonal
of the blocks $\mathbf{A}_{2}$ (equation \ref{eq:Aq_block}), which are located at the corners 
of $\mathbf{A}$ (equation \ref{eq:BTTB_A}).
To verify this, let us take the matrix indices associated with these elements.
They are $(i, j)$ = $(1, 9)$, $(2, 10)$, $(3, 11)$, $(4, 12)$, $(9, 1)$, $(10, 2)$, 
$(11, 3)$ and $(12, 4)$. By using these matrix indices, it is easy to verify that all
of them produce the same grid indices $l(i)$, $l(j)$, $k(i)$ and $k(j)$ 
(equations \ref{eq:l-x-oriented} and \ref{eq:k-x-oriented}) as those associated with
the element defined by $i = 2$ and $j = 10$. Consequently, all of them produce
elements $a_{ij}$ (equation \ref{eq:aij-regular-grids}) having the same value.
Besides, it is also easy to verify that all these matrix indices produce the same block
indices $q = 2$ (equation \ref{eq:q-x-oriented}) and $p = 0$ (equation \ref{eq:p-x-oriented})
and the same block element $a^{q}_{p}$ (equation \ref{eq:aqp_equiv_aij}).
By repeating this procedure for all elements $a_{ij}$, $i = 1, \dots, 12$, $j = 1, \dots, 12$, 
forming the matrix $\mathbf{A}$ (equation \ref{eq:predicted-data-vector}) obtained from our 
$x$-oriented grid (Figure 1a), we can verify that
\begin{equation}
\mathbf{A} = \begin{bmatrix}
\mathbf{A}_{0} & \mathbf{A}_{1} & \mathbf{A}_{2} \\
\mathbf{A}_{1} & \mathbf{A}_{0} & \mathbf{A}_{1} \\
\mathbf{A}_{2} & \mathbf{A}_{1} & \mathbf{A}_{0}
\end{bmatrix} \quad ,
\label{eq:A-x-oriented-example}
\end{equation}
where $\mathbf{A}_{q}$ (equation \ref{eq:Aq_block}), $q = 0, \dots, Q -1$, $Q = N_{y}$, 
are symmetric Toeplitz matrices given by:
\begin{equation}
\mathbf{A}_{q} = \begin{bmatrix}
a^{q}_{0} & a^{q}_{1} & a^{q}_{2} & a^{q}_{3} \\
a^{q}_{1} & a^{q}_{0} & a^{q}_{1} & a^{q}_{2} \\
a^{q}_{2} & a^{q}_{1} & a^{q}_{0} & a^{q}_{1} \\
a^{q}_{3} & a^{q}_{2} & a^{q}_{1} & a^{q}_{0}
\end{bmatrix} \quad ,
\label{eq:Aq-x-oriented}
\end{equation}
with elements $a^{q}_{p}$ (equation \ref{eq:aqp_equiv_aij}) defined by 
$p = 0, \dots, P - 1$, $P = N_{x}$.

This procedure can also be used to verify that the matrix $\mathbf{A}$ 
(equation \ref{eq:predicted-data-vector}) obtained
from the $y$-oriented grid illustrated in Figure 1b is given by
\begin{equation}
\mathbf{A} = \begin{bmatrix}
\mathbf{A}_{0} & \mathbf{A}_{1} & \mathbf{A}_{2} & \mathbf{A}_{3} \\
\mathbf{A}_{1} & \mathbf{A}_{0} & \mathbf{A}_{1} & \mathbf{A}_{2} \\
\mathbf{A}_{2} & \mathbf{A}_{1} & \mathbf{A}_{0} & \mathbf{A}_{1} \\
\mathbf{A}_{3} & \mathbf{A}_{2} & \mathbf{A}_{1} & \mathbf{A}_{0}
\end{bmatrix} \quad ,
\label{eq:A-y-oriented-example}
\end{equation}
where $\mathbf{A}_{q}$ (equation \ref{eq:Aq_block}), $q = 0, \dots, Q - 1$, $Q = N_{x}$, 
are symmetric Toeplitz matrices given by:
\begin{equation}
\mathbf{A}_{q} = \begin{bmatrix}
a^{q}_{0} & a^{q}_{1} & a^{q}_{2} \\
a^{q}_{1} & a^{q}_{0} & a^{q}_{1} \\
a^{q}_{2} & a^{q}_{1} & a^{q}_{0}
\end{bmatrix} \quad ,
\label{eq:Aq-y-oriented}
\end{equation}
with elements $a^{q}_{p}$ (equation \ref{eq:aqp_equiv_aij}) defined by 
$p = 0, \dots, P - 1$, $P = N_{y}$.

These examples (equations \ref{eq:A-x-oriented-example}--\ref{eq:Aq-y-oriented}) show 
that the entire $N \times N$ BTTB matrix $\mathbf{A}$ 
(equations \ref{eq:predicted-data-vector} and \ref{eq:BTTB_A}) 
can be defined by using only the elements 
forming its first column (or row). Notice that this column contains the gravitational effect 
produced by a single equivalent source at all $N$ observation points.

\subsection{BTTB matrix-vector product}

The matrix-vector product $\tensor{A} \hat{\mathbf{p}}^{k}$ (equation 
\ref{eq:delta_p_k_fast_eqlayer}) required by the fast equivalent-layer 
technique \citep{siqueira-etal2017} accounts for most of its total computation time 
and can cause RAM memory shortage when large data sets are used.
This computational load can be drastically lessen by exploring the well-defined structure of 
matrix $\mathbf{A}$ (equation \ref{eq:predicted-data-vector}) for the particular case in which 
its elements $a_{ij}$ are defined by equation \ref{eq:aij-regular-grids}. 
In this case, $\mathbf{A}$ is a symmetric BTTB matrix (equations \ref{eq:BTTB_A} and 
\ref{eq:A-x-oriented-example}--\ref{eq:Aq-y-oriented}) and the predicted data vector 
$\mathbf{d}(\mathbf{p})$ (equation \ref{eq:predicted-data-vector}) can be efficiently
computed by using the two-dimensional Discrete Fourier Transform (DFT).
To do this, let us first rewrite $\mathbf{d}(\mathbf{p})$ and
$\mathbf{p}$ (equation \ref{eq:predicted-data-vector}) as the following partitioned vectors:
\begin{equation}
\mathbf{d}(\mathbf{p}) = \begin{bmatrix}
\mathbf{d}_{0}(\mathbf{p}) \\
\vdots \\
\mathbf{d}_{Q - 1}(\mathbf{p})
\end{bmatrix}_{N \times 1}
\label{eq:predicted-data-vector-partitioned}
\end{equation}
and
\begin{equation}
\mathbf{p} = \begin{bmatrix}
\mathbf{p}_{0} \\
\vdots \\
\mathbf{p}_{Q - 1}
\end{bmatrix}_{N \times 1} \quad ,
\label{eq:parameter-vector-partitioned}
\end{equation}
where $\mathbf{d}_{q}(\mathbf{p})$ and $\mathbf{p}_{q}$, $q = 0, \dots, Q - 1$,
are $P \times 1$ vectors. Notice that $q$ is the block index defined by equations 
\ref{eq:q-x-oriented} and \ref{eq:q-y-oriented}, $Q$ defines the number of blocks
$\mathbf{A}_{q}$ (equation \ref{eq:Aq_block}) forming $\mathbf{A}$ (equation \ref{eq:BTTB_A}) 
and $P$ defines the number of elements forming each block $\mathbf{A}_{q}$.
Then, by using the partitioned vectors 
(equations \ref{eq:parameter-vector-partitioned} and \ref{eq:predicted-data-vector-partitioned}) 
and remembering that $N = QP$, we define the auxiliary linear system
\begin{equation}
\mathbf{w} = \mathbf{C} \mathbf{v} \: ,
\label{eq:w_Cv}
\end{equation}
where
\begin{equation}
\mathbf{w} = \begin{bmatrix}
\mathbf{w}_{0} \\
\vdots \\
\mathbf{w}_{Q - 1} \\
\mathbf{0}_{2N \times 1}
\end{bmatrix}_{4N \times 1} \quad ,
\label{eq:w-vector}
\end{equation}
\begin{equation}
\mathbf{w}_{q} = \begin{bmatrix}
\mathbf{d}_{q}(\mathbf{p}) \\
\mathbf{0}_{P \times 1}
\end{bmatrix}_{2P \times 1}
\label{eq:wq-vector} \quad ,
\end{equation}
\begin{equation}
\mathbf{v} = \begin{bmatrix}
\mathbf{v}_{0} \\
\vdots \\
\mathbf{v}_{Q - 1} \\
\mathbf{0}_{2N \times 1}
\end{bmatrix}_{4N \times 1} \quad ,
\label{eq:v-vector}
\end{equation}
and
\begin{equation}
\mathbf{v}_{q} = \begin{bmatrix}
\mathbf{p}_{q} \\
\mathbf{0}_{P \times 1}
\end{bmatrix}_{2P \times 1}
\label{eq:vq-vector} \quad ,
\end{equation}
with $\mathbf{d}_{q}(\mathbf{p})$ and $\mathbf{p}_{q}$ defined by
equations \ref{eq:predicted-data-vector-partitioned} and 
\ref{eq:parameter-vector-partitioned}, respectively.
Finally $\mathbf{C}$ (equation \ref{eq:w_Cv}) is a 
$4N \times 4N$ symmetric Block Circulant matrix with Circulant Blocks (BCCB) 
\citep[][ p. 184]{davis1979}.
What follows shows a step-by-step description of how we use the auxiliary 
system (equation \ref{eq:w_Cv}) to compute the matrix-vector product 
$\tensor{A} \hat{\mathbf{p}}^{k}$ (equation \ref{eq:delta_p_k_fast_eqlayer}) in 
a computationally efficient way by exploring the structure of matrix $\mathbf{C}$.
We show that this approach, that have been used in potential-field methods
\citep[e.g.,][]{zhang-wong2015, zhang-etal2016, qiang_etal2019}, is actually a fast 
convolution method \citep[e.g.,][ p. 213]{vanloan1992}.

Matrix $\mathbf{C}$ (equation \ref{eq:w_Cv})
is circulant blockwise, formed by $Q \times Q$ blocks, where
each block $\mathbf{C}_{q}$, $q = 0, \dots, Q-1$, is a $2P \times 2P$ circulant matrix. 
Similarly to the BTTB matrix $\mathbf{A}$ (equations \ref{eq:BTTB_A} and 
\ref{eq:A-x-oriented-example}--\ref{eq:Aq-y-oriented}), 
the index $q$ defines the block diagonal where $\mathbf{C}_{q}$ lies 
within $\mathbf{C}$. This index varies from $0$, at the main diagonal, to $Q - 1$, at
the corners of $\mathbf{C}$. Additionally, the blocks lying 
above the main diagonal are equal to those located below.
It is well-known that a circulant matrix can be defined by properly downshifting 
its first column \citep[][ p. 206]{vanloan1992}. Hence, the BCCB matrix $\mathbf{C}$ 
(equation \ref{eq:w_Cv}) can be obtained from its 
first column of blocks, which is given by
\begin{equation}
\left[\mathbf{C} \right]_{(0)} = 
	\begin{bmatrix}
	\mathbf{C}_{0} \\
	\vdots \\
	\mathbf{C}_{Q-1} \\
	\mathbf{0} \\
	\mathbf{C}_{Q-1} \\
	\vdots \\
	\mathbf{C}_{1}
	\end{bmatrix}_{4N \times 2P} \: ,
	\label{eq:C-first-column-blocks}
\end{equation}
where $\mathbf{0}$ is a $2P \times 2P$ matrix of zeros. Similarly, each block 
$\mathbf{C}_{q}$, $q = 0, \dots, Q-1$, can be obtained by downshifting its first 
column
\begin{equation}
\mathbf{c}^{q}_{0} = 
	\begin{bmatrix}
		a^{q}_{0} \\
		\vdots \\
		a^{q}_{P-1} \\
		0 \\
		a^{q}_{P-1} \\
		\vdots \\
		a^{q}_{1}
	\end{bmatrix}_{2P \times 1} \: ,
	\label{eq:Cq-first-column}
\end{equation}
where $a^{q}_{p}$ (equation \ref{eq:aqp_equiv_aij}), $p = 0, \dots, P-1$, are the elements 
forming the block $\mathbf{A}_{q}$ (equations \ref{eq:Aq_block} and 
\ref{eq:A-x-oriented-example}--\ref{eq:Aq-y-oriented}).
The downshift can be thought off as permutation that pushes the components of a column vector 
down one notch with wraparound \citep[][ p. 20]{golub-vanloan2013}.
To illustrate this operation, consider our $y$-oriented grid illustrated in Figure 1b. 
In this case, the resulting 
BCCB matrix $\mathbf{C}$ (equation \ref{eq:w_Cv}) is given by 
\begin{equation}
\mathbf{C} =
\begin{bmatrix}
	\mathbf{C_{0}} & \mathbf{C_{1}} & \mathbf{C_{2}} & \mathbf{C_{3}} & \mathbf{0}     & \mathbf{C_{3}} & \mathbf{C_{2}} & \mathbf{C_{1}} \\
	\mathbf{C_{1}} & \mathbf{C_{0}} & \mathbf{C_{1}} & \mathbf{C_{2}} & \mathbf{C_{3}} & \mathbf{0}     & \mathbf{C_{3}} & \mathbf{C_{2}} \\
	\mathbf{C_{2}} & \mathbf{C_{1}} & \mathbf{C_{0}} & \mathbf{C_{1}} & \mathbf{C_{2}} & \mathbf{C_{3}} & \mathbf{0}     & \mathbf{C_{3}} \\
	\mathbf{C_{3}} & \mathbf{C_{2}} & \mathbf{C_{1}} & \mathbf{C_{0}} & \mathbf{C_{1}} & \mathbf{C_{2}} & \mathbf{C_{3}} & \mathbf{0}     \\
	\mathbf{0}     & \mathbf{C_{3}} & \mathbf{C_{2}} & \mathbf{C_{1}} & \mathbf{C_{0}} & \mathbf{C_{1}} & \mathbf{C_{2}} & \mathbf{C_{3}} \\
	\mathbf{C_{3}} & \mathbf{0}     & \mathbf{C_{3}} & \mathbf{C_{2}} & \mathbf{C_{1}} & \mathbf{C_{0}} & \mathbf{C_{1}} & \mathbf{C_{2}} \\
	\mathbf{C_{2}} & \mathbf{C_{3}} & \mathbf{0}     & \mathbf{C_{3}} & \mathbf{C_{2}} & \mathbf{C_{1}} & \mathbf{C_{0}} & \mathbf{C_{1}} \\
	\mathbf{C_{1}} & \mathbf{C_{2}} & \mathbf{C_{3}} & \mathbf{0}     & \mathbf{C_{3}} & \mathbf{C_{2}} & \mathbf{C_{1}} & \mathbf{C_{0}}
\end{bmatrix},
\label{eq:C-y-oriented}
\end{equation}
where each block $\mathbf{C}_{q}$, $q = 0, 1, 2$, is represented as follows 
\begin{equation}
\tensor{C}_{q} =
\begin{bmatrix}
	a^{q}_{0} & a^{q}_{1} & a^{q}_{2} & 0         & a^{q}_{2} & a^{q}_{1} \\
	a^{q}_{1} & a^{q}_{0} & a^{q}_{1} & a^{q}_{2} & 0         & a^{q}_{2} \\
	a^{q}_{2} & a^{q}_{1} & a^{q}_{0} & a^{q}_{1} & a^{q}_{2} & 0         \\
	0         & a^{q}_{2} & a^{q}_{1} & a^{q}_{0} & a^{q}_{1} & a^{q}_{2} \\
	a^{q}_{2} & 0         & a^{q}_{2} & a^{q}_{0} & a^{q}_{0} & a^{q}_{1} \\
	a^{q}_{1} & a^{q}_{2} & 0         & a^{q}_{2} & a^{q}_{1} & a^{q}_{0}
\end{bmatrix}
\label{eq:Cq-y-oriented}
\end{equation}
in terms of the block elements $a^{q}_{p}$ (equation \ref{eq:aqp_equiv_aij}).
Similar matrices are obtained for our $x$-oriented grid illustrated in Figure 1a.

BCCB matrices are diagonalized by the two-dimensional unitary DFT 
\citep[][ p. 185]{davis1979}. It means that $\mathbf{C}$ (equation \ref{eq:w_Cv}) 
satisfies 
\begin{equation}
\mathbf{C} = 
\left(\mathbf{F}_{2Q} \otimes \mathbf{F}_{2P} \right)^{\ast} 
\boldsymbol{\Lambda}
\left(\mathbf{F}_{2Q} \otimes \mathbf{F}_{2P} \right) \: ,
\label{eq:C-diagonalized}
\end{equation}
where the symbol ``$\otimes$" denotes the Kronecker product \citep{neudecker1969},
$\mathbf{F}_{2Q}$ and $\mathbf{F}_{2P}$ are the $2Q \times 2Q$ and $2P \times 2P$ 
unitary DFT matrices \citep[][ p. 31]{davis1979}, respectively, the superscritpt 
``$\ast$" denotes the complex conjugate and $\boldsymbol{\Lambda}$ is a 
$4QP \times 4QP$ diagonal matrix containing the eigenvalues of $\mathbf{C}$.
By substituting equation \ref{eq:C-diagonalized} in the auxiliary system 
(equation \ref{eq:w_Cv}) and premultiplying both
sides of the result by $\left(\mathbf{F}_{2Q} \otimes \mathbf{F}_{2P} \right)$, 
we obtain
\begin{equation}
\boldsymbol{\Lambda} \left(\mathbf{F}_{2Q} \otimes \mathbf{F}_{2P} \right) 
\mathbf{v} = \left(\mathbf{F}_{2Q} \otimes \mathbf{F}_{2P} \right) 
\mathbf{w} \: .
\label{eq:vec-DFT-system}
\end{equation}
Now, by applying the $vec$-operator to both sides of equation \ref{eq:vec-DFT-system} 
(see the details in Appendix A), we obtain:
\begin{equation}
\mathbf{F}_{2Q}^{\ast} \left[ 
\mathbf{L} \circ \left(\mathbf{F}_{2Q} \, \mathbf{V} \, \mathbf{F}_{2P} \right) 
\right] \mathbf{F}_{2P}^{\ast} = \mathbf{W} \: ,
\label{eq:DFT-system}
\end{equation}
where ``$\circ$'' denotes the Hadamard product \citep[][ p. 298]{horn_johnson1991} and 
$\mathbf{L}$, $\mathbf{V}$ and $\mathbf{W}$ are $2Q \times 2P$ matrices obtained 
by rearranging, along their rows, the elements forming the diagonal of matrix 
$\boldsymbol{\Lambda}$, vector $\mathbf{v}$ and vector $\mathbf{w}$, respectively.
The left side of equation \ref{eq:DFT-system} contains the two-dimensional 
Inverse Discrete Fourier Transform (IDFT) of the term in brackets, which in turn
represents the Hadamard product of matrix $\mathbf{L}$ (equation \ref{eq:left_side_DFT_system_3})
and the two-dimensional DFT of matrix $\mathbf{V}$ (equation \ref{eq:left_side_DFT_system_1}).
Matrix $\mathbf{L}$ contains the eigenvalues 
of $\boldsymbol{\Lambda}$ (equation \ref{eq:C-diagonalized}) and can be 
efficiently computed by using only the first column of the BCCB matrix 
$\mathbf{C}$ (equation \ref{eq:w_Cv}) (see the details in Appendix B).
Here, we evaluate equation \ref{eq:DFT-system} and compute matrix $\mathbf{L}$
by using the two-dimensional Fast Fourier Transform (2D FFT).
This approach is a fast two-dimensional discrete convolution \citep[e.g.,][ p. 213]{vanloan1992}.

At each iteration $k$th of the fast equivalent-layer technique, 
(equation \ref{eq:delta_p_k_fast_eqlayer}), we efficiently compute 
$\mathbf{A} \hat{\mathbf{p}}^{k} = \mathbf{d}(\hat{\mathbf{p}}^{k})$ by following 
the steps below:

\begin{itemize}
\item[\textbf{(1)}] Use equation \ref{eq:aij-regular-grids} to compute the first column 
of each block $\mathbf{A}_{q}$ (equation \ref{eq:Aq_block}), $q = 0, \dots, Q-1$, forming 
the BTTB matrix $\mathbf{A}$ (equation \ref{eq:BTTB_A});

\item[\textbf{(2)}] Rearrange the first column of $\mathbf{A}$ according to equations 
\ref{eq:C-first-column-blocks} and \ref{eq:Cq-first-column} to obtain the
first column $\mathbf{c}_{0}$ of the BCCB matrix $\mathbf{C}$ (equation \ref{eq:w_Cv});

\item[\textbf{(3)}] Rearrange $\mathbf{c}_{0}$ along the rows of matrix $\mathbf{G}$ and
use the 2D FFT to compute matrix $\mathbf{L}$ (equation \ref{eq:DTF_G});

\item[\textbf{(4)}] Rearrange the parameter vector $\hat{\mathbf{p}}^{k}$ 
(equation \ref{eq:predicted-data-vector}) in its partitioned form 
(equation \ref{eq:parameter-vector-partitioned}) to define the auxiliary vector 
$\mathbf{v}$ (equation \ref{eq:v-vector});

\item[\textbf{(5)}] Rearrange $\mathbf{v}$ to obtain matrix $\mathbf{V}$, use the 2D FFT 
to compute its DFT and evaluate the left side of equation \ref{eq:DFT-system-preliminary};

\item[\textbf{(6)}] Use the 2D FFT to compute the IDFT of the result obtained at item (5) and 
obtain the matrix $\mathbf{W}$ (equation \ref{eq:DFT-system});

\item[\textbf{(7)}] Use the $vec$-operator (equation \ref{eq:vec-operator}) and equations 
\ref{eq:w-vector} and \ref{eq:wq-vector} to rearrange $\mathbf{W}$ and obtain the predicted 
data vector $\mathbf{d}(\hat{\mathbf{p}}^{k})$.

\end{itemize}


\subsection{Computational performance}


The number of flops (floating-point operations) necessary to estimate the 
$N \times 1$ parameter vector $\mathbf{p}$ in the fast equivalent-layer technique 
\citep{siqueira-etal2017} is
\begin{equation}
f_{0} = N^{it} (3N + 2N^{2}) \; ,
\label{eq:float_fast_eqlayer}
\end{equation}
where $N^{it}$ is the number of iterations. In this equation, the term $2N^2$ is associated 
with the matrix-vector product $\mathbf{A} \hat{\mathbf{p}}^{k}$ (equation 
\ref{eq:delta_p_k_fast_eqlayer}) and accounts for most of the computational complexity 
of this method.
Our method replace this matrix-vector product by three operations: 
one DFT, one Hadammard product and one IDFT involving $2Q \times 2P$ matrices 
(left side of equation \ref{eq:DFT-system}). 
The Hadamard product requires six times $4N$ flops, $N = QP$, because the entries are 
complex numbers.
We consider that a DFT/IDFT requires $\kappa \, 4N \log_{2}(4N)$ flops to be computed via 2D FFT, 
where $\kappa$ is a constant depending on the algorithm. 
Then, the resultant flops count of our method is given by:
\begin{equation}
f_{1} = N^{it} \left[ 27N + \kappa \, 8N \log_{2}(4N) \right] \: .
\label{eq:float_bccb}
\end{equation}
Figure \ref{fig:float} shows the flops counts $f_{0}$ and $f_{1}$ (equations \ref{eq:float_fast_eqlayer}
and \ref{eq:float_bccb}) associated with the fast equivalent-layer technique \citep{siqueira-etal2017} and 
our method, respectively, as a function of the number $N$ of observation points. 
We considered a fixed number of $N^{it} = 50$ iterations and $\kappa = 5$ (equation \ref{eq:float_bccb}),
which is compatible with a radix-2 FFT \citep[][ p. 16]{vanloan1992}.
As we can see, the number of flops is drastically decreased in our method.

Another advantage of our method is concerned with the real $N \times N$ matrix $\mathbf{A}$ 
(equations \ref{eq:predicted-data-vector} and \ref{eq:BTTB_A}).
In the fast equivalent-layer technique, the full matrix 
is computed once and stored during the entire iterative process.
On the other hand, our method computes only one column of $\mathbf{A}$ 
and uses it to compute the complex $2Q \times 2P$ matrix $\mathbf{L}$ 
(equation \ref{eq:DTF_G}) via 2D FFT, which is stored during all iterations.
Table \ref{tab:RAM-usage} shows the RAM memory usage needed to store the 
full matrix $\mathbf{A}$, a single column of $\mathbf{A}$ and the full matrix 
$\mathbf{L}$. The quantities were computed for different numbers of observations $N$. 
Notice that $N = 1\,000\,000$ observations require nearly $7.6$ Terabytes of RAM memory 
to store the whole matrix $\mathbf{A}$.

Figure \ref{fig:time_fast_eqlayer_bccb} compares the running time of the fast equivalent-layer technique 
\citep{siqueira-etal2017} and of our method, considering a constant number of iterations $N^{it} = 50$. 
We used a PC with an Intel Core i7 4790@3.6GHz processor and 16 GB of RAM memory.
The computational efficiency of our approach is significantly superior to that of the 
fast equivalent-layer technique for a number of observations $N$ greater than $10\,000$. 
We could not perform this comparison with a number of observations greater than $22\,500$
due to limitations of our PC in storing the full matrix $\mathbf{A}$.
Figure \ref{fig:time_bccb} shows the running time of our method with a number of observations 
up to  $25$ millions. 
These results shows that, while the running time of our method is $\approx 30.9$ seconds for 
$N = 1\,000\,000$, the fast equivalent-layer technique spends $\approx 46.8$ seconds for $N = 22\,500$.
\section{Synthetic tests}


In this section, we  investigate the effectiveness of using the properties of BTTB and BCCB matrices 
(equation \ref{eq:w_Cv})  to solve, at each iteration, the forward modeling (the matrix-vector product 
$\mathbf{A} \hat{\mathbf{p}}^k$)  required in the fast equivalent-layer method proposed by \citet{siqueira-etal2017}. 
We simulated three sources whose horizontal projections are shown in Figure \ref{fig:synthetic_data} as black lines. 
These sources are two vertical prisms with density contrasts of $0.35\, \mathrm{g/cm^3}$ (upper-left prism) and 
$0.4\, \mathrm{g/cm^3}$ (upper-right prism) and a sphere with radius of $1\,000$ m with density contrast of 
$-0.5\, \mathrm{g/cm^3}$. Figure \ref{fig:synthetic_data}  shows the vertical component of gravity field generated 
by these sources contaminated with additive pseudorandom Gaussian noise with zero mean and standard deviation of 
$0.015$ mGal.
 
The advantage of using the structures of BTTB and BCCB matrices to compute forward modeling in  the fast-equivalent 
layer method \cite[]{siqueira-etal2017} is grounded on the use of regular grids of data and equivalent sources. 
Hence, we created $10\,000$ observation points regularly spaced in a grid of $100 \times 100$ at $100$ m height. 
We also set a grid of equivalent point masses, each one directly beneath each observation points, located at 
$300$ m deep.  Figures \ref{fig:classic_fast_val}a and \ref{fig:bccb_fast_val}a  show the fitted gravity data 
obtained, respectively, by the fast equivalent layer method and by our modified form of this method that computes 
the forward modeling using equation \ref{eq:w_Cv}. The corresponding residuals (Figures \ref{fig:classic_fast_val}b 
and \ref{fig:bccb_fast_val}b), defined as the difference between the observed (Figure  \ref{fig:synthetic_data}) 
and fitted gravity data (Figures \ref{fig:classic_fast_val}a and \ref{fig:bccb_fast_val}a), show means close to 
zero  and standard deviations of 0.0144 mGal.  Therefore, Figures \ref{fig:classic_fast_val} and 
\ref{fig:bccb_fast_val} show that \citet{siqueira-etal2017} method and our modified version of this method produced 
virtually the same results. This excellent agreement is confirmed in Figures \ref{fig:delta_comparison} and 
\ref{fig:delta_rho} which shows that there are virtually no differences, respectively,  in the fitted data presented 
in Figures \ref{fig:classic_fast_val}b and \ref{fig:bccb_fast_val}b and in the estimated mass distributions within 
the equivalent layers (not shown) yielded by both \citet{siqueira-etal2017} method and our modification of this method. 
These results (Figures \ref{fig:delta_comparison} and \ref{fig:delta_rho})  show that computing the matrix-vector 
product ($\mathbf{A} \hat{\mathbf{p}}^k$), required in the forward modeling, by means of embedding the BTTB matrix into 
a BCCB matrix (equation \ref{eq:w_Cv}) yields practically the same result as the one produced by computing this 
matrix-vector product  with a full matrix $\mathbf{A}$ as used in \citet{siqueira-etal2017}.

We perform two forms of processing the gravity data (Figure \ref{fig:synthetic_data}) through the equivalent layer 
technique: the upward (Figure \ref{fig:upward_comparison}) and the downward (Figure \ref{fig:downward_comparison}) 
continuations. The upward height is 300 m and the downward  is at 50 m.   Either in the upward continuation 
(Figure \ref{fig:upward_comparison}) or in the downward continuation (Figure \ref{fig:downward_comparison}), the 
continued gravity data using the fast equivalent layer proposed by \citet{siqueira-etal2017} 
(Figures \ref{fig:upward_comparison}a and \ref{fig:downward_comparison}a) are in close agreement with those produced 
by our modification of \citet{siqueira-etal2017} method (Figures \ref{fig:upward_comparison}b and 
\ref{fig:downward_comparison}b). The residuals (Figures \ref{fig:upward_comparison}c and \ref{fig:downward_comparison}c)
quantify this agreement since their means and standard deviations are close to zero in both continued gravity data using 
both methods.  All the continued gravity data shown here (Figures \ref{fig:upward_comparison} and 
\ref{fig:downward_comparison}) agree with the true ones (not shown). The most striking feature of these upward or 
the downward continuations concerns the total computation time. The computation time spent by our method is 
approximately $1\,500$ times faster than \citet{siqueira-etal2017} method. 
\section{Application to field data}

We applied our method to airborne gravity data from Caraj{\'a}s, north of Brazil, which were provided by 
the Geological Survey of Brazil (CPRM). The data were collected along $131$ north-south flight lines separated 
by $3$ km and $29$ east-west tie lines separated by $12$ km.
This data set was divided in two different areas, collected in different times, having samples spacing of 
$7.65$ m and $15.21$ m along the lines, totalizing  $4\,353\,428$ observation points at a fixed height 
of $900$ m ($z_{1} = -900$ m). 
The gravity data were interpolated (Figure \ref{fig:carajas_real_data}) into a regularly spaced grid of 
$500 \times 500$ observation points ($N = 250\,000$) with a grid spacing of $716.9311$ m north-south and 
$781.7387$ m east-west.

To apply our method, we set an equivalent layer at $z_{0} = 300$ m. 
Figure \ref{fig:carajas_gz_predito_val}a shows the predicted data obtained with our method after 
$N^{it} = 50$ iterations.
The residuals (Figure \ref{fig:carajas_gz_predito_val}b), defined as the difference between the observed 
(Figure \ref{fig:carajas_real_data}) and predicted (Figure \ref{fig:carajas_gz_predito_val}a) data, show a 
very good data fit with mean close to zero ($0.0003$ mGal) and small standard deviation ($0.105$ mGal), 
which corresponds to approximately $0.1$ \% of the maximum amplitude of the gravity data.
By using the estimated mass distribution (not shown), we performed an upward-continuation of the 
observed gravity data to a horizontal plane located $5\,000$ m above. Figure \ref{fig:up2000_carajas_500x500} 
shows a very consistent upward-continued gravity data, with a clear attenuation of the short 
wavelengths. By using our approach, the processing of the $250\,000$ observations took only 
$0.216$ seconds.
\section{Conclusions}

We show that the sensitivity matrix associated with the equivalent layer technique 
for processing gravity data has a BTTB structure when the following conditions are
satisfied: (i) the observed data are disposed at a regular grid on a horizontal plane
and (ii) the equivalent sources (point masses) are placed at a constant depth, one directly 
beneath each observation point.
By exploring the BTTB structure of the sensitivity matrix, we formulate the forward modeling as a 
fast FFT convolution that requires only one column of the sensitivity matrix 
and propose an efficient approach for optimizing the computational time of an 
iterative method called fast equivalent-layer technique.
This iterative method is grounded on the excess mass constraint and does not demand the solution of 
linear systems.
Our approach greatly reduces the number of flops and the RAM memory necessary to estimate a 
2D mass distribution within a planar equivalent layer that fits the observed gravity data. 
For example, the number of flops of the fast equivalent-layer technique is reduced by two orders 
of magnitude when processing one million of observations by using our approach. 
Traditionally, such amount of data impractically requires $7.6$ Terabytes of RAM memory to handle 
the full sensitivity matrix whereas our method takes only $61.035$ Megabytes.
This drastic reduction comes from the fact that, by using only one column of the sensitivity matrix,
our method is able to compute the gravity data produced by the whole equivalent layer with a single 
point mass via FFT.
We have successfully applied the proposed method to compute the upward/downward continuation of 
synthetic gravity data. Application to field data over the Caraj{\'a}s Province, north of Brazil, 
confirms the potential of our approach in processing a gravity data set with $250\,000$ observations 
in approximately $0.2$ seconds. Further studies need to be carried out in order to apply our 
method to process other potential-field data.
%\section{Acknowledgements}

%This study was financed by the brazilian agencies CAPES (in the form of a scholarship), FAPERJ (grant n.$^{\circ}$ E-26 202.729/2018) and CNPq (grant n.$^{\circ}$ 308945/2017-4).

% Tables and figures
\tabl{RAM-usage}{Comparison between the system RAM memory usage needed to store the full 
matrix $\mathbf{A}$ (equations \ref{eq:predicted-data-vector} and \ref{eq:BTTB_A}), a single 
column of $\mathbf{A}$ and the full matrix $\mathbf{L}$ (equation \ref{eq:DTF_G}). The quantities 
are represented in megabyte (MB), for different total number of observations $N$. We consider 
that the elements of $\mathbf{A}$ and $\mathbf{L}$ occupy, respectively, 8 and 16 bytes each 
in computer memory.
\label{tab:RAM-usage}
}{
	\begin{center}
		\begin{tabular}[]{|l|c|c|c|}
			\hline
			\textbf{$N$} & \textbf{Full matrix $\mathbf{A}$} & \textbf{Single column of $\mathbf{A}$} 
			& \textbf{Full matrix $\mathbf{L}$}\\
			\hline 
			$100$ & 0.0763 & 0.000763 & 0.00610\\
			\hline
			$400$ & 1.22 & 0.0031 & 0.0248\\
			\hline
			$2\,500$ & 48 & 0.0191 & 0.1528\\
			\hline
			$10\,000$ & 763 & 0.0763 & 0.6104\\
			\hline
			$40\,000$ & 12\,207 & 0.305 & 2.4416 \\
			\hline
			$250\,000$ & 476\,837 & 1.907 & 15.3 \\
			\hline
			$500\,000$ & 1\,907\,349 & 3.815 & 30.518 \\
			\hline
			$1\,000\,000$ & 7\,629\,395 & 7.629 & 61.035 \\
			\hline
		\end{tabular}
	\end{center} 
}
\renewcommand{\figdir}{Fig} % figure directory

\plot{Figure1}{width=\textwidth}{
	{Schematic representation of an $N_{x} \times N_{y}$ regular grid of points (black dots) defined by 
	$N_{x} = 4$ and $N_{y} = 3$. The grids are oriented along the (a) $x$-axis and (b) $y$-axis. The grid 
	coordinates are $x_{k}$ and $y_{k}$, where the $k = 1, \dots, N_{x}$ and $l = 1, \dots, N_{y}$ are 
	called the grid indices. The insets show the grid indices $k$ and $l$.}
	\label{fig:methodology}
}

% Computational performance

\plot{Figure2}{width=\textwidth}{
	{Comparison between the number of flops (equations \ref{eq:float_fast_eqlayer} and \ref{eq:float_bccb}) 
	associated with the fast equivalent-layer technique \citep{siqueira-etal2017} and our method, 
	for $N$ varying from $5\,000$ to $1\,000\,000$. All values are computed with $N^{it} = 50$ 
	iterations and $\kappa = 5$.}
	\label{fig:float}
}

\plot{Figure3}{width=10cm}{
	{Comparison between the running time of the fast equivalent-layer technique \citep{siqueira-etal2017} and our method.
	The values were obtained for $N^{it} = 50$ iterations.}
	\label{fig:time_fast_eqlayer_bccb}
}

\plot{Figure4}{width=10cm}{
	{Running time of our method for a number of observations $N$ up to $25$ millions. 
	The values were obtained for $N^{it} = 50$ iterations.}
	\label{fig:time_bccb}
}

% Application to synthetic data

\plot{Figure5}{width=13.5cm}{
	{Application to synthetic data. 
	Noise-corrupted gravity data (in color map) produced by three synthetic sources: 
	a sphere with density contrast $-0.5\, \mathrm{g/cm^3}$ and 
	two rectangular prisms with density contrasts $0.35\, \mathrm{g/cm^3}$ (upper-left body) and 
	$0.4\, \mathrm{g/cm^3}$ (upper-right body). The black lines represent the horizontal projection 
	of the sources.}
	\label{fig:synthetic_data}
}

\plot{Figure6}{width=8cm}{
	{Application to synthetic data. 
	(a) Predicted data produced by the fast equivalent-layer technique \citep{siqueira-etal2017}. 
	(b) Residuals between the simulated (Figure \ref{fig:synthetic_data}) and 
	predicted (shown in a) data. The mean and standard deviation are $8.264 \times 10^{-7}$ and $0.0144$ mGal,
	respectively.}
	\label{fig:classic_fast_val}
}

\plot{Figure7}{width=8cm}{
	{Application to synthetic data. 
	(a) Predicted data produced by our method. 
	(b) Residuals between the simulated (Figure \ref{fig:synthetic_data}) and 
	predicted (shown in a) data. The mean and standard deviation are $8.264 \times 10^{-7}$ and $0.0144$ mGal,
	respectively.}
	\label{fig:bccb_fast_val}
}

\plot{Figure8}{width=13.5cm}{
	{Application to synthetic data. 
	Difference between the predicted data produced by the fast equivalent-layer technique 
	(Figure \ref{fig:classic_fast_val}a) and by our method (Figure \ref{fig:bccb_fast_val}a).}
	\label{fig:delta_comparison}
}

\plot{Figure9}{width=13.5cm}{
	{Application to synthetic data. 
	Difference between the estimated mass distribution within the equivalent layers produced by 
	the fast equivalent-layer technique (not shown) and by our method (not shown).}
	\label{fig:delta_rho}
}

\plot{Figure10}{width=5cm}{
	{Application to synthetic data. 
	The upward-continued gravity data obtained with (a) the fast equivalent-layer technique 
	\citep{siqueira-etal2017} and (b) our method. (c) Residuals between 
	panels a and b with mean $-5.938 \times 10^{-18}$ mGal and standard deviation $8.701 \times 10^{-18}$ mGal. 
	The total computation times spent by the fast equivalent-layer technique and our method 
	are $7.62026$ and $0.00834$ seconds, respectively.}
	\label{fig:upward_comparison}
}

\plot{Figure11}{width=5cm}{
	{Application to synthetic data. 
	The downward-continued gravity data obtained with (a) the fast equivalent-layer technique 
	\citep{siqueira-etal2017} and (b) our method. (c) Residuals between 
	panels a and b with mean $5.914 \times 10^{-18}$ mGal and standard deviation $9.014 \times 10^{-18}$ mGal. 
	The total computation times spent by the fast equivalent-layer technique and our method 
	are $7.59654$ and $0.00547$ seconds, respectively.}
	\label{fig:downward_comparison}
}

% Application to field data

\plot{Figure12}{width=13.5cm}{
	{Application to field data over the Caraj{\'a}s Province, Brazil. 
	Observed gravity data on a regular grid of $500 \times 500$ points, totaling $N = 250,000$ 
	observations. The inset shows the study area (blue rectangle) which covers the southeast part of 
	the state of Par{\'a}, north of Brazil.}
	\label{fig:carajas_real_data}
}

\plot{Figure13}{width=8cm}{
	{Application to field data over the Caraj{\'a}s Province, Brazil. 
	(a) Predicted data produced by our method. (b) Residuals between the observed  
	(Figure \ref{fig:carajas_real_data}) and the predicted data (panel a), with mean 
	$0.000292$ mGal and standard deviation of $0.105$ mGal.}
	\label{fig:carajas_gz_predito_val}
}

\plot{Figure14}{width=13.5cm}{
	{Application to field data over the Caraj{\'a}s Province, Brazil. 
	The upward-continued gravity data obtained with our method $5\,000$ m above the 
	observed data (Figure \ref{fig:carajas_real_data}).
	The total computation time for processing of the $250,000$ observations was $0.216$ seconds.}
	\label{fig:up2000_carajas_500x500}
}

% Appendices
\append{Computations with the 2D DFT}


In the present Appendix, we deduce equation \ref{eq:DFT-system}
by using the row-ordered $vec$-operator (here designated simply as $vec$-operator).
This equation can be efficiently computed by using the 2D 
fast Fourier Transform. 
This operator was implicitly used by \citet[][ p. 31]{jain1989} to 
show the relationship between Kronecker products and separable 
transformations. The $vec$-operator defined here 
transforms a matrix into a column vector by stacking its rows. 

Let $\mathbf{M}$ be an arbitrary $N \times M$ matrix given by:
\begin{equation}
\mathbf{M} = \begin{bmatrix}
\mathbf{m}^{\top}_{1} \\ 
\vdots \\
\mathbf{m}^{\top}_{N}
\end{bmatrix} \: ,
\label{eq:matrix-M}
\end{equation}
where $\mathbf{m}_{i}$, $i = 1, \dots, N$, are $M \times 1$ vectors containing 
the rows of $\mathbf{M}$.
The elements of this matrix can be rearranged into a column vector by using the
$vec$-operator \citep[][ p. 31]{jain1989} as follows:
\begin{equation}
vec \left( \mathbf{M} \right) = \begin{bmatrix}
\mathbf{m}_{1} \\
\vdots \\
\mathbf{m}_{N}
\end{bmatrix}_{NM \times 1} \: .
\label{eq:vec-operator}
\end{equation}
This rearrangement is known as lexicographic ordering \citep[][ p. 150]{jain1989}.

Two important properties of the $vec$-operator (equation \ref{eq:vec-operator}) 
are necessary to us. 
To define the first one, consider an 
$N \times M$ matrix $\mathbf{H}$ given by
\begin{equation}
\mathbf{H} = \mathbf{P} \circ \mathbf{Q} \: ,
\label{eq:matrix-H}
\end{equation}
where $\mathbf{P}$ and $\mathbf{Q}$ are arbitrary $N \times M$ matrices and 
``$\circ$" represents the Hadamard product \citep[][ p. 298]{horn_johnson1991}.
By applying the $vec$-operator to $\mathbf{H}$ (equation \ref{eq:matrix-H}), 
it can be shown that
\begin{equation}
vec \left( \mathbf{H} \right) = 
vec \left( \mathbf{P} \right) \circ vec \left( \mathbf{Q} \right) \: .
\label{eq:vec-matrix-H}
\end{equation}
To define the second important property of $vec$-operator, 
consider an $N \times M$ matrix $\mathbf{S}$ defined by 
the separable transformation \citet[][ p. 31]{jain1989}:
\begin{equation}
\mathbf{S} = \mathbf{P \, M \, Q} \: ,
\label{eq:matrix-S}
\end{equation}
where $\mathbf{P}$ and $\mathbf{Q}$ are arbitrary $N \times N$ and $M \times M$ 
matrices, respectively.
By implicitly applying the $vec$-operator to 
the $\mathbf{S}$ (equation \ref{eq:matrix-S}), 
\citet[][ p. 31]{jain1989} showed that:
\begin{equation}
vec \left( \mathbf{S} \right) = 
\left( \mathbf{P} \otimes \mathbf{Q}^{\top} \right) 
vec \left( \mathbf{M} \right) \: ,
\label{eq:vec-matrix-S}
\end{equation}
where ``$\otimes$" denotes the Kronecker product \citep{neudecker1969}.
It is important to stress the difference between equation \ref{eq:vec-matrix-S}
and that presented by \citet{neudecker1969}, which is more commonly found in 
the literature.
While that equation uses a $vec$-operator that transforms a matrix into a column 
vector by stacking its columns, equation \ref{eq:vec-matrix-S} 
uses the $vec$-operator defined by equation \ref{eq:vec-operator}, which 
transforms a matrix into a column vector by stacking its rows.

Now, let us deduce equation \ref{eq:DFT-system} by 
using the above-defined properties (equation \ref{eq:vec-matrix-H}
and \ref{eq:vec-matrix-S}).
We start calling attention to the right side of equation \ref{eq:vec-DFT-system}.
Consider that vector $\mathbf{w}$ (equation \ref{eq:vec-DFT-system}) 
is obtained by applying the $vec$-operator (equation \ref{eq:vec-operator}) to a matrix 
$\mathbf{W}$, whose 2D DFT $\tilde{\mathbf{W}}$ is represented by the 
following separable transformation \citep[][ p. 146]{jain1989}:
\begin{equation}
\tilde{\mathbf{W}} = \mathbf{F}_{2Q} \, \mathbf{W} \, \mathbf{F}_{2P} \: ,
\label{eq:2D-DFT-W}
\end{equation}
where $\mathbf{F}_{2Q}$ and $\mathbf{F}_{2P}$ are the $2Q \times 2Q$ and $2P \times 2P$ 
unitary DFT matrices. 
Using equation \ref{eq:vec-matrix-S} and the symmetry of unitary DFT 
matrices, we rewrite the right side of equation \ref{eq:vec-DFT-system} 
as follows:
\begin{equation}
vec \left( \tilde{\mathbf{W}} \right) = 
\left( \mathbf{F}_{2Q} \otimes \mathbf{F}_{2P} \right) 
vec \left( \mathbf{W} \right) \: .
\label{eq:right_side_DFT_system_1}
\end{equation}
Similarly, consider that $\mathbf{v}$ (equation \ref{eq:vec-DFT-system}) 
is obtained by applying the $vec$-operator (equation \ref{eq:vec-operator}) to a matrix 
$\mathbf{V}$, whose 2D DFT (equation \ref{eq:2D-DFT-W}) is 
represented by $\tilde{\mathbf{V}}$. Using equation \ref{eq:vec-matrix-S} and the symmetry 
of unitary DFT matrices, we can rewrite the 
left side of equation \ref{eq:vec-DFT-system} as follows:
\begin{equation}
\boldsymbol{\Lambda} \, vec \left( \tilde{\mathbf{V}} \right) = 
\boldsymbol{\Lambda}
\left( \mathbf{F}_{2Q} \otimes \mathbf{F}_{2P} \right) 
vec \left( \mathbf{V} \right) \: .
\label{eq:left_side_DFT_system_1}
\end{equation}
Note that both sides of equation \ref{eq:left_side_DFT_system_1}
are defined as the product of the diagonal matrix $\boldsymbol{\Lambda}$ (equation \ref{eq:C-diagonalized}) 
and a vector. In this case, the matrix-vector product can be conveniently replaced by
\begin{equation}
\boldsymbol{\lambda} \circ vec \left( \tilde{\mathbf{V}} \right) = 
\boldsymbol{\lambda} \circ
\left( \mathbf{F}_{2Q} \otimes \mathbf{F}_{2P} \right) 
vec \left( \mathbf{V} \right) \: ,
\label{eq:left_side_DFT_system_2}
\end{equation}
where $\boldsymbol{\lambda}$ is a $4QP \times 1$ vector containing the diagonal of 
$\boldsymbol{\Lambda}$ (equation \ref{eq:C-diagonalized}).
Then, consider that $\boldsymbol{\lambda}$ is obtained by applying the $vec$-operator 
(equation \ref{eq:vec-operator}) to a $2Q \times 2P$ matrix $\mathbf{L}$, we can use 
equations \ref{eq:vec-matrix-H} and \ref{eq:vec-matrix-S} to rewrite equation 
\ref{eq:left_side_DFT_system_2} as follows:
\begin{equation}
vec \left( \mathbf{L} \circ \tilde{\mathbf{V}} \right) = 
vec \left[ \mathbf{L} \circ 
\left( \mathbf{F}_{2Q} \, \mathbf{V} \, \mathbf{F}_{2P} \right) 
\right] \: .
\label{eq:left_side_DFT_system_3}
\end{equation}
Equations \ref{eq:2D-DFT-W}, \ref{eq:right_side_DFT_system_1} and 
\ref{eq:left_side_DFT_system_3} show that equation \ref{eq:vec-DFT-system}
is obtained by applying the $vec$-operator to 
\begin{equation}
\mathbf{L} \circ \left( \mathbf{F}_{2Q} \, \mathbf{V} \, \mathbf{F}_{2P} \right) = 
\mathbf{F}_{2Q} \, \mathbf{W} \, \mathbf{F}_{2P} \: .
\label{eq:DFT-system-preliminary}
\end{equation}
Finally, we premultiply both sides of equation \ref{eq:DFT-system-preliminary} by 
$\mathbf{F}_{2Q}^{\ast}$ and then postmultiply both sides of the result by 
$\mathbf{F}_{2P}^{\ast}$ to deduce equation \ref{eq:DFT-system}.
\append{The eigenvalues of $\mathbf{C}$}

In the present Appendix, we show how to efficiently compute matrix $\mathbf{L}$
(equations \ref{eq:left_side_DFT_system_3}, \ref{eq:DFT-system-preliminary} 
and \ref{eq:DFT-system}) by using only the first column of the BCCB matrix
$\mathbf{C}$ (equation \ref{eq:w_Cv}).

We need first premultiply both sides of equation \ref{eq:C-diagonalized}
by $\left(\mathbf{F}_{2Q} \otimes \mathbf{F}_{2P} \right)$ to obtain
\begin{equation}
\left(\mathbf{F}_{2Q} \otimes \mathbf{F}_{2P} \right) \mathbf{C} = 
\boldsymbol{\Lambda}
\left(\mathbf{F}_{2Q} \otimes \mathbf{F}_{2P} \right) \: .
\label{eq:C-diagonalized2}
\end{equation}
From equation \ref{eq:C-diagonalized2}, we can easily show that 
\citep[][ p. 77]{chan-jin2007}:
\begin{equation}
	\left(\mathbf{F}_{2Q} \otimes \mathbf{F}_{2P} \right) \, 
	\mathbf{c}_{0} = \frac{1}{\sqrt{4QP}} \, \boldsymbol{\lambda} \: ,
	\label{eq:DFT_C_column}
\end{equation}
where $\mathbf{c}_{0}$ is a $4QP \times 1$ vector representing the first column of 
$\mathbf{C}$ (equation \ref{eq:w_Cv}) and 
$\boldsymbol{\lambda}$ (equation \ref{eq:left_side_DFT_system_2}) is the $4QP \times 1$ 
vector that contains the diagonal of matrix $\boldsymbol{\Lambda}$ (equation \ref{eq:C-diagonalized}) 
and is obtained by applying the $vec$-operator (equation \ref{eq:vec-operator}) to matrix $\mathbf{L}$.
Now, let us conveniently consider that $\mathbf{c}_{0}$ is obtained by applying the $vec$-operator 
to a $2Q \times 2P$ matrix $\mathbf{G}$.
Using this matrix, the property of the $vec$-operator for separable transformations 
(equation \ref{eq:matrix-S}) and the symmetry of unitary DFT matrices, equation \ref{eq:DFT_C_column} 
can be rewritten as follows
\begin{eqnarray}
	\mathbf{F}_{2Q} \, \mathbf{G} \, \mathbf{F}_{2P} = 
	\frac{1}{\sqrt{4QP}} \, \mathbf{L} \: .
	\label{eq:DTF_G}
\end{eqnarray}
This equation shows that the eigenvalues of the BCCB matrix $\mathbf{C}$ 
(equation \ref{eq:w_Cv}), forming the rows of $\mathbf{L}$,
are obtained by computing the two-dimensional DFT of matrix $\mathbf{G}$,
which contains the elements forming the first column of the BCCB matrix 
$\mathbf{C}$ (equation \ref{eq:w_Cv}).

\bibliographystyle{seg}  % style file is seg.bst
\bibliography{references}

\end{document}
