\begin{abstract}

We have developed an efficient and very fast equivalent-layer technique for gravity data processing by modifying an 
iterative method grounded on excess mass constraint that does not require the solution of linear systems. 
Taking advantage of the symmetric Block-Toeplitz Toeplitz-block (BTTB) structure of the sensitivity matrix, that raises 
when regular grids of observation points and equivalent sources (point masses) are used to set up a fictitious 
equivalent layer, we have developed an algorithm which greatly reduces the number of flops and RAM memory necessary 
to estimate a 2D mass distribution over the equivalent layer. The structure of symmetric BTTB matrix consists of 
the elements of the first column of the sensitivity matrix, which in turn can be embedded into a symmetric 
Block-Circulant Circulant-Block (BCCB) matrix. Likewise, only the first column of the BCCB matrix is needed 
to reconstruct the full sensitivity matrix completely. From the first column of BCCB matrix, its eigenvalues 
can be calculated using the 2D Fast Fourier Transform (2D FFT), which can be used to readily 
compute the matrix-vector product of the forward modeling in the fast equivalent-layer technique. As a result, 
our method is efficient to process very large datasets using either fine- or mid-grid meshes. The larger the 
dataset, the faster and more efficient our method becomes compared to the available equivalent-layer techniques. 
Synthetic tests demonstrate the ability of our method to satisfactorily upward- and downward-continuing the 
gravity data. For example, while our method requires 26.8 seconds to run one million of observations, the fast 
equivalent-layer technique required 48.3 seconds to run 22,500 observations. Test with real data from Caraj{\'a}s, 
Brazil, shows its applicability to process very large dataset at low computational cost.

\end{abstract}