\begin{abstract}

We develop an efficient and very fast equivalent-layer technique for 
gravity data processing by modifying an iterative method grounded on an excess mass 
constraint that does not require the solution of linear systems. 
Taking advantage of the symmetric Block-Toeplitz Toeplitz-Block (BTTB) structure 
of the sensitivity matrix, that arises when regular grids of observation points 
and equivalent sources (point masses) are used to set up a fictitious 
equivalent layer, we develop an algorithm that greatly reduces the computational 
complexity and RAM memory necessary to estimate a 2D mass distribution over the 
equivalent layer. 
The structure of symmetric BTTB matrix consists of the elements of the first column of 
the sensitivity matrix, which in turn can be embedded into a symmetric 
Block-Circulant Circulant-Block (BCCB) matrix. 
Likewise, only the first column of the BCCB matrix is needed 
to reconstruct the full sensitivity matrix completely. From the first column of 
BCCB matrix, its eigenvalues can be calculated using the 2D Fast Fourier Transform 
(2D FFT), which can be used to readily compute the matrix-vector product of the 
forward modeling in the fast equivalent-layer technique. 
As a result, our method is efficient for processing very large datasets. 
Tests with synthetic data demonstrate the ability of our method to satisfactorily upward-
and downward-continuing gravity data.
Our results show very small border effects and noise amplification compared to those 
produced by the classical approach in the Fourier domain.
Besides, they show that while the running time of our method is $\approx 30.9$ seconds for 
processing $N = 1,000,000$ observations, the fast equivalent-layer technique spent 
$\approx 46.8$ seconds with $N = 22,500$.
A test with field data from Caraj{\'a}s Province, Brazil, illustrates the low computational 
cost of our method to process a large data set composed of $N = 250,000$ observations.

\end{abstract}