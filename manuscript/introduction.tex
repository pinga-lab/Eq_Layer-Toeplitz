\section{Introduction}

The equivalent layer is a well-known technique for processing potential-field data in applied geophysics since 1960. 
It comes from potential theory as a mathematical solution of the Laplace's equation, in the region above the sources, 
by using the Dirichlet boundary condition \citep{kellogg1929}.
This theory states that any potential-field data produced by an arbitrary 3D physical-property distribution can be 
exactly reproduced by a fictitious layer located at any depth and having a continuous 2D physical-property  
distribution. In practical situations, the layer is approximated by a finite set of sources (e.g., point masses or 
dipoles) and their physical properties are estimated by solving a linear system of equations that yield an 
acceptable potential-field data fit. These fictitious sources are called equivalent sources.

Many previous works have used the equivalent layer as a processing technique of potential-field data. 
\citet{dampney1969} used the equivalent-layer technique for gridding and for computing the upward 
continuation of the potential-field data. \citet{cordell1992} and \citet{mendonca-silva1994} used it for 
interpolating and gridding potential-field data. \citet{emilia1973}, \citet{hansen-miyazaki1984} and 
\citet{li-oldenburg2010} used it for upward continuation of the potential-field data. \citet{silva1986}, 
\citet{leao-silva1989}, \citet{guspi-novara2009}, and \citet{oliveirajr-etal2013} used it for reducing the 
magnetic data to the pole. \citet{boggs-dransfield2004} used it for combining multiple data sets and 
\citet{barnes-lumley2011} for gradient-data processing.

The classic equivalent-layer formulation consists in estimating the physical-property distribution within a layer, 
composed by a set of equivalent sources, by solving a linear system of equations formed by harmonic functions 
(e.g.,the inverse of the distance between the observation point and the equivalent source). When these observation 
points and equivalent sources are regularly spaced, a Toeplitz system arises. Toeplitz systems are well-known in 
many branches of science as in (1) mathematics, for solving partial and ordinary diferential equations 
\citep[e.g.,][]{lin-etal2003}; (2) image processing \citep[e.g.,][]{chan-etal1999} and; (3) computational 
neuroscience \citep[e.g.,][]{wray-green1994}. \citet{jin2003} and \citet{chan-jin2007} give many examples of 
applications for Toeplitz systems.

In potential-field methods, the properties of Toeplitz system were used for downward continuation 
\citep{zhang-etal2016} and for 3D gravity-data inversion using a 2D multilayer model \citep{zhang-wong2015}. 
In the particular case of gravity data, the kernel generates a linear system with a matrix known as symmetric 
Block-Toeplitz Toeplitz-Block (BTTB).

INCLUIR \citet{qiang_etal2019}

A wide variety of applications in mathematics and engineers that fall into Toeplitz systems propelled the development 
of a large variety of  methods for solving them. 
%Direct methods were conceived by \citet{levinson1946} and by \citet{trench1964}. Currently the conjugate gradient 
%is used in most cases.
\citet{grenander-szego1984} noticed that a circulant matrix can be diagonalized by taking the Fast Fourier Trasform 
(FFT) of its first column, making it possible to calculate the matrix-vector product and solve the system with 
low computational cost \citep{strang-aarikka1986, olkin1986}. \citet{chan-jin2007} show some preconditioners to embedd 
the Toeplitz and BTTB matrices into, respectively, circulant matrices and Block-Circulant Circulant-Block (BCCB) by 
solving the system applying the conjugate gradient method.

Although the use of the equivalent-layer technique increased over the last decades, one of the biggest problem is still 
its high computational cost for processing large-data sets. \citet{siqueira-etal2017} developed a computationally 
efficient scheme for processing gravity data. This scheme does not solve a linear system, instead uses an iterative 
process that corrects the physical-property distribution over the equivalent layer by adding mass corrections that 
are proportional to the gravity residual. Although efficient, the method presented by \citet{siqueira-etal2017} requires, 
at each iteration,  the full computation of the forward problem to guarantee the convergence of the algorithm. The 
time spent on forward modeling accounts for most of the total computation time of their method.

We propose the use of BTTB and BCCB matrices properties to efficiently  solve the forward modeling in method of 
\citet{siqueira-etal2017}, resulting in a very faster parameter estimation and the possibility to use very large 
datasets. Here, we show how the system memory (RAM) usage can be drastically decreased by calculating only the 
first column of the BTTB matrix and embedding into a BCCB matrix. Using the Szeg\"{o} theorema  combined with 
\cite{strang-aarikka1986}, the matrix-vector product can be accomplished with very low cost, reducing in some 
orders of magnitude the number of operations required to complete the process. We present synthetic tests to validate 
our proposal and real field data from Caraj\'as, Brazil to demonstrate its applicability.