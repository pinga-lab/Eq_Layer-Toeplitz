\section{Introduction}

The equivalent layer is a well-known technique for processing potential-field data in applied geophysics 
since the 60's. 
It comes from potential theory as a mathematical solution of the Laplace's equation, in the region above the
sources, by using the Dirichlet boundary condition \citep{kellogg1929}.
This theory states that any potential-field data produced by an arbitrary 3D physical-property distribution can 
be exactly reproduced by a fictitious layer located at any depth and having a continuous 2D physical-property  
distribution. In practical situations, the layer is approximated by a finite set of sources (e.g., point masses 
or dipoles) and their physical properties are estimated by solving a linear system of equations that yield an 
acceptable potential-field data fit. These fictitious sources are called equivalent sources.

Many previous works have used the equivalent layer to perform different potential-field data 
transformations such as gridding \citep[e.g.,][]{dampney1969, cordell1992, mendonca-silva1994},
upward/downward continuation \citep[e.g.,][]{emilia1973, hansen-miyazaki1984, 
li-oldenburg2010}, reduction to the pole \citep[e.g.,]{silva1986, leao-silva1989, guspi-novara2009, 
oliveirajr-etal2013}, combining multiple data sets \citep[e.g.,][]{boggs-dransfield2004} and 
gradient data processing \citep[e.g.,][]{barnes-lumley2011}.

Although the use of the equivalent-layer technique increased over the last decades, one of the biggest 
problems is still its high computational cost for processing large-data sets. 
This problem propelled several studies to improve the computational efficiency of the 
equivalent layer technique. 
\citet{leao-silva1989} developed a fast method for processing a regular grid of potential-field data.
The method consists in estimating an equivalent layer which exactly reproduces the potential-field data 
within a small data window. The data window is shifted over the whole gridded data set in a procedure similar 
to a discrete convolution. 
The equivalent layer extends beyond the moving-data window and is located at a depth between two and six 
times the grid spacing of the observations. For each data window, the equivalent layer is estimated by 
solving an underdetermined linear system.
After estimating an equivalent layer, the transformed-potential field is computed only at the center of the 
moving-data window. The use of a small moving-data window greatly reduces the total number of floating-point 
operations (\textit{flops}) and RAM memory storage. The computational efficiency of this method relies on the 
strategy of constructing the equivalent layer by successively solving small linear systems instead of solving 
just one large linear system for the entire equivalent layer. 
\citet{mendonca-silva1994} also followed the strategy of solving successive small linear systems for 
constructing an equivalent layer. Their method is based on the  equivalent-data concept, which 
consists in determining a subset of all potential-field data (named equivalent-data set), such that the 
interpolating surface that fits the chosen subset also automatically fits all remaining data. 
The equivalent data set is obtained by iteratively introducing the potential-field observation with the 
greatest residual in the preceding iteration. By applying to the interpolation problem, the method is 
optimized by approximating dot products by the discrete form of an analytic integration that can be 
evaluated with less computational effort. According to the authors, the equivalent-data set is usually 
smaller than the total number of potential-field observations, leading to computational savings. 
The authors also pointed out that the computational efficiency of the method depends on the number of 
equivalent data. If the potential-field anomaly is nonsmooth, the number of equivalent data can be large and 
the method will be less efficient than the classical approach.

By following a different strategy, \citet{li-oldenburg2010} developed a rapid method that transforms the 
dense sensitivity matrix associated with the linear system into a sparse one by using a wavelet technique. 
After obtaining a sparse representation of the sensitivity matrix, those authors estimate the 
physical-property distribution within the equivalent layer by using an overdetermined formulation. 
Those authors pointed out that, given the sparse representation, their method reduces the computational time 
required for solving the linear system by as many as two orders of magnitude if compared with the same 
formulation using a dense matrix. 
\citet{barnes-lumley2011} followed a similar strategy and transformed the dense sensitivity matrix 
into a sparse one. However, differently from \citet{li-oldenburg2010}, their method operates in the space 
domain by grouping equivalent sources far from an observation point into blocks with average physical 
property. This procedure aims at reducing the memory storage and achieving computational efficiency 
by solving the transformed linear system with a weighted-least-squares conjugate-gradient algorithm.
Notice that, instead of constructing the equivalent layer by solving successive small linear systems, 
these last two methods first transform the large linear system into a sparse one and then take advantage of 
this sparseness.

\citet{oliveirajr-etal2013} developed a fast method based on the reparameterization of the physical-property 
distribution within the equivalent layer. Those authors divided the equivalent layer into a regular grid of 
equivalent-source windows inside which the physical-property distribution is described by bivariate 
polynomial functions. By using this polynomial representation, the inverse problem for estimating the 
equivalent layer is posed in the space of the total number of polynomial coefficients within all 
equivalent-source windows instead of in the space of the total number of equivalent sources. According to 
\citet{oliveirajr-etal2013}, the computational efficiency of their method relies on the fact that the total 
number of polynomial coefficients needed to describe the physical property distribution within the 
equivalent layer is generally much smaller than the number of equivalent sources, leading to a very smaller 
linear system to be solved. Those authors could verify that the total number of \textit{flops} needed for 
building and solving the linear inverse problem of estimating the total number of polynomial coefficients can 
be reduced by as many as three and four orders of magnitude, respectively, if compared with the same inverse 
problem of estimating the physical property of each equivalent source via Cholesky decomposition.

There is another class of methods that iteratively estimates the physical-property distribution within 
the equivalent layer without solving linear systems.
The method presented by \citet{cordell1992} and after generalized by \citet{guspi-novara2009} updates 
the physical property of the sources, which are located below each potential-field data, using a 
procedure that removes the maximum residual between the observed and predicted data.
\citet{xia-sprowl1991} and \citet{xia-etal1993} developed 
fast iterative schemes for updating the physical-property distribution
within the equivalent layer in the wavenumber and space domains, respectively.
Grounded on excess mass constraint, \cite{siqueira-etal2017} developed an iterative scheme 
starting with a mass distribution within the equivalent layer that is proportional to observed gravity data.
Then, their method iteratively adds mass corrections that are proportional to the gravity residuals.
The total number of \textit{flops} required by these iterative methods for estimating the physical-property 
distribution within the equivalent layer depends on the total number of iterations, however this number is 
generally much smaller than the total number of \textit{flops} required to solve a large-scaled linear system. 
Generally, the most computational expensive step in each iteration of these methods is the forward problem 
of calculating the potential-field data produced by the equivalent layer.

In the present work, we show that the sensitivity matrix associated with a planar equivalent layer 
of point masses has a very well-defined structure called Block-Toeplitz Toeplitz-Block (BTTB) for 
the case in which (i) the observed gravity data is located on a regularly spaced grid at constant 
height and (ii) there is one point mass directly beneath each observation point.
By using this property, we propose an efficient algorithm based on fast FFT convolution 
\citep[e.g.,][ p. 207]{vanloan1992} for computing the forward problem at each iteration of 
the fast equivalent-layer technique proposed by \citet{siqueira-etal2017}.
Our method uses the gravitational effect produced by a single point mass to compute the 
effect produced by the whole equivalent layer, which results in a drastic reduction 
not only in the number of flops, but also in the RAM memory usage of the fast equivalent-layer technique.
This technique have been successfully used in potential-field methods for 
3D gravity inversion \citep{zhang-wong2015}, downward continuation 
\citep{zhang-etal2016} and 3D magnetic modeling \citep{qiang_etal2019}.
Tests with synthetic and field data illustrate the good performance of our method in processing 
large gravity data sets.



