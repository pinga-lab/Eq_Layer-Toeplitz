\section{Conclusions}
By exploring the BTTB structure of the sensitivity matrix in the gravity data processing, we have proposed a new 
efficient approach for calculating the gravity-data forward modeling required in the iterative fast equivalent-layer 
technique grounded on excess mass constraint that does not demand  the solution of linear systems.  Its efficiency 
requires the use of regular grids of observations and equivalent sources (point masses). Our algorithm greatly reduces 
the number of  flops necessary to estimate a 2D mass distribution within the equivalent layer that fits the observed 
gravity data. For example, when processing one million observations the number of flops is reduced in 104 times. 
Traditionally, such amount of data impractically requires 7.6 Terabytes of RAM memory to handle the full sensitivity 
matrix. Rather, in our method, this matrix takes 61.035 Megabytes of RAM memory only.

Our method takes advantage of the symmetric BTTB system that arises when processing a harmonic function and considering 
that either the observations or the sources of the interpretative model (point of masses over the equivalent layer) 
are distributed on regular grids. Symmetric BTTB matrices can be stored by using only a single column and can be 
embedded into a symmetric BCCB matrix, which in turn also only needs a single column. 
This means that only a single column  of the sensitivity matrix needs to be calculated.  Once the jth column of the 
sensitivity matrix represents the influence of the jth source (jth point mass) has on the predicted data, our method 
only requires a single point mass that set up the equivalent layer to compute the gravity-data forward modeling.

Using the 2D FFT, it is possible to calculate the eigenvalues of BCCB matrices which can
be used to compute a matrix-vector product (gravity-data forward modeling) in a very low computational cost. 
We have successfully applied the proposed method to upward (or downward) synthetic gravity data. Testing on field 
data from the Caraj{\'a}s Province, north of Brazil, confirms the potential of our approach in upward-continuing 
gravity data with $250\,000$ observations in  about 0.2 seconds.  Our method allows, in future research, applying 
the equivalent layer-technique for processing and interpreting massive data set such as collected in continental 
and global scales studies.