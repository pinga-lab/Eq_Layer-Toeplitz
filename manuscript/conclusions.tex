\section{Conclusions}

We show that the sensitivity matrix associated with the equivalent layer technique 
for processing gravity data has a BTTB structure when the following conditions are
satisfied: (i) the observed data are disposed at a regular grid on a horizontal plane
and (ii) the equivalent sources (point masses) are placed at a constant depth, one directly 
beneath each observation point.
By exploring the BTTB structure of the sensitivity matrix, we formulate the forward modeling as a 
fast FFT convolution that requires only one column of the sensitivity matrix 
and propose an efficient approach for optimizing the computational time of an 
iterative method called fast equivalent-layer technique.
This iterative method is grounded on the excess mass constraint and does not demand the solution of 
linear systems.
Compared to the fast equivalent-layer technique, 
our approach greatly reduces the number of flops and the RAM memory necessary to estimate a 
2D mass distribution within a planar equivalent layer that fits the observed gravity data. 
For example, the number of flops is reduced by two orders 
of magnitude when processing one million of observations by using our approach. 
Traditionally, such amount of data impractically requires $7.6$ Terabytes of RAM memory to handle 
the full sensitivity matrix whereas our method takes only $61.035$ Megabytes.
This drastic reduction comes from the fact that, by using only one column of the sensitivity matrix,
our method is able to compute the gravity data produced by the whole equivalent layer with a single 
point mass via FFT.
We have successfully applied the proposed method to compute the upward/downward continuation of 
synthetic gravity data. 
Our results show that our method produces considerably smaller border effects and 
noise amplification compared to the classical Fourier approach, 
which computes the Fourier transform of the observed data.
Application to field data over the Caraj{\'a}s Province, north of Brazil, 
confirms the potential of our approach in processing an airborne gravity data set with $250\,000$ observations in approximately $0.2$ seconds. 
Our method can also be applied to process large marine gravity data sets. 
Further studies need to be carried out in order to apply our method to process 
other potential-field data.