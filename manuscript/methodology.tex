\section{Methodology}

\subsection{Equivalent-layer technique for gravity data processing}

Let $d^{o}_{i}$ be the observed gravity data at
the point $(x_{i}, y_{i}, z_{i})$, $i = 1, ..., N$, of a local Cartesian
system with $x$ axis pointing to north, the $y$ axis pointing to east and 
the $z$ axis pointing downward.
Let us consider an equivalent layer composed by a set of $N$ point masses 
(equivalent sources) over a layer located at depth $z_0$ ($z_0 >z_i$) and whose 
$x$- and $y$- coordinates of each point mass coincides with the corresponding coordinates 
of the observation directly above.
There is a linear relationship that maps the unknown mass distribution onto the gravity 
data given by
\begin{equation}
\mathbf{d}(\mathbf{p}) = \mathbf{A} \mathbf{p} \: ,
\label{eq:predicted-data-vector}
\end{equation}
where $\mathbf{d}$ is an $N \times 1$ vector whose $i$th element is the predicted gravity 
data at the $i$th point ($x_i$,$y_i$,$z_i$), $\mathbf{p}$ is the unknown $N \times 1$ 
parameter vector whose $j$th element $p_j$  is the mass of the $j$th equivalent source 
(point mass) at the $j$th Cartesian coordinates ($x_j$,$y_j$,$z_0$) and $\mathbf{A}$ 
is an $N \times N$  sensitivity matrix whose $ij$th element is given by 
\begin{equation}
a_{ij}= \frac{c_{g} \, G \, (z_{0} - z_{i})}{\left[(x_{i} - x_{j})^{2} +
(y_{i} - y_{j})^{2} +	(z_{i} - z_{0})^{2} \right]^{\frac{3}{2}}} \; ,
\label{eq:aij}
\end{equation}
where $G$ is the Newton's gravitational constant and $c_{g} = 10^{5}$ 
transforms from $\mathrm{m/s^2}$ to mGal.
Notice that the sensitivity matrix depends on the $i$th coordinate of the observation 
and the $j$th coordinate of the equivalent source. For convenience, we designate 
these coordinates as \textit{matrix coordinates} and the indices $i$ and $j$ as 
\textit{matrix indices}.
In the classical equivalent-layer technique, we estimate the regularized parameter vector 
from the observed gravity data $\mathbf{d}^{o}$ by
\begin{equation}
\hat{\mathbf{p}} = \left( \mathbf{A}^{\top}\mathbf{A} + 
\mu \, \mathbf{I} \right)^{-1}
\mathbf{A}^{\top} \mathbf{d}^{o} \: .
\label{eq:estimated-p-parameter-space}
\end{equation}

\subsection{Fast equivalent-layer technique}

\citet{siqueira-etal2017} developed an iterative least-squares method to estimate the mass 
distribution over the equivalent layer based on the excess of mass and the positive correlation 
between the observed gravity data and the masses on the equivalent layer. They showed 
that the fast equivalent-layer technique has a better computational efficiency than the 
classical equivalent layer approach (equation \ref{eq:estimated-p-parameter-space}) if the 
dataset is greater than at least 200 observation points, even using a large number of iterations.

Considering one equivalent source (point mass) directly beneath each observation point, 
the iteration of the \citeauthor{siqueira-etal2017}'s~(\citeyear{siqueira-etal2017}) method 
starts by an initial approximation of mass distribution given by
\begin{equation}
\hat{\mathbf{p}}^0 = \tilde{\mathbf{A}}^{-1} \mathbf{d}^{o} \: ,
\label{eq:p0_fast_eqlayer}
\end{equation}
where $\tilde{\mathbf{A}}^{-1}$ is an $N \times N$ diagonal matrix with elements
\begin{equation}
\tilde{a}_{ii}^{-1} = \frac{\Delta s_i}{(2 \pi \, G \, c_g)} \: ,
\label{eq:aii_tilde_inv_fast_eqlayer}
\end{equation}
where $\Delta s_i$ is the $i$th element of surface area located at the $i$th horizontal 
coordinates $x_i$ and $y_i$ of the $i$th observation.
At the $k$th iteration, the masses of the equivalent sources are updated by
\begin{equation}
\hat{\mathbf{p}}^{k+1} = \hat{\mathbf{p}}^{k} + \mathbf{\Delta} \hat{\mathbf{p}}^{k} \: ,
\label{eq:p_k+1_fast_eqlayer}
\end{equation}
where the mass correction is given by
\begin{equation}
\mathbf{\Delta} \hat{\mathbf{p}}^{k+1} = \tilde{\mathbf{A}}^{-1} (\mathbf{d}^{o} - \mathbf{A} \hat{\mathbf{p}}^{k}) \: .
\label{eq:delta_p_k_fast_eqlayer}
\end{equation}

At the $k$th iteration of \citeauthor{siqueira-etal2017}'s~(\citeyear{siqueira-etal2017}) method, 
the matrix-vector product 
$\tensor{A} \hat{\mathbf{p}}^{k} = \mathbf{d}(\hat{\mathbf{p}}^{k})$ must be calculated to get a 
new residual $\mathbf{d^0} - \tensor{A} \hat{\mathbf{p}}^{k}$, which represents a bottleneck. 
Considering the limitation of 16 Gb of RAM memory in our system, we could run the 
\citeauthor{siqueira-etal2017}'s~(\citeyear{siqueira-etal2017}) method only up to 22 500 observation points;
Otherwise, it is costly and can be prohibitive in terms of RAM 
memory to maintain such operation.

\subsection{Structure of matrix $\mathbf{A}$ for regular grids}

Consider that the observed data are located on an $N_{x} \times N_{y}$ regular grid of
points regularly spaced by $\Delta x$ and $\Delta y$ along the $x$ and $y$ directions,
respectively, on a horizontal plane defined by the constant vertical coordinate $z_{1} < z_{0}$. 
As a consequence, a given pair of matrix coordinates $(x_{i}, y_{i})$, defined by the matrix index 
$i$, $i = 1, \dots, N = N_{x} N_{y}$, is equivalent to a pair of coordinates $(x_{k}, y_{l})$
given by:
\begin{equation}
x_{i} \equiv x_{k} = x_{1} + \left[ k(i) - 1 \right] \, \Delta x \: , 
\label{eq:xi}
\end{equation}
and
\begin{equation}
y_{i} \equiv y_{l} = y_{1} + \left[ l(i) - 1 \right] \, \Delta y \: ,
\label{eq:yi}
\end{equation}
where $k(i)$ and $l(i)$ are integer functions of the matrix index $i$.
These equations can also be used to define the matrix coordinates 
$x_{j}$ and $y_{j}$ associated with the $j$th equivalent source,
$j = 1, \dots, N = N_{x}N_{y}$. In this case, the integer functions
are evaluated by using the index $j$ instead of $i$.
For convenience, we designate $x_{k}$ and $y_{l}$ as \textit{grid coordinates}
and the indices $k$ and $l$ as \textit{grid indices}, which are computed with
the integer functions.

The integer functions assume different forms depending on the 
orientation of the regular grid of data.
Consider the case in which the grid is oriented along the
$x$-axis (Figure \ref{fig:methodology}a). For convenience, we designate these grids as 
$x$-\textit{oriented grids}. For them, we have the following integer functions:
\begin{equation}
i(k, l) = (l - 1) \, N_{x} + k \quad ,
\label{eq:i-x-oriented}
\end{equation}
\begin{equation}
l(i) = \Bigg\lceil \frac{i}{N_{x}} \Bigg\rceil
\label{eq:l-x-oriented}
\end{equation}
and
\begin{equation}
k(i)  = i - \Bigg\lceil \frac{i}{N_{x}} \Bigg\rceil N_{x} + N_{x} \quad ,
\label{eq:k-x-oriented}
\end{equation}
where $\lceil \cdot \rceil$ denotes the ceiling function \citep[][ p. 67]{graham-etal1994}.
These integer functions are defined in terms of the matrix index $i$, but they can 
be defined in the same way by using the index $j$.
Figure \ref{fig:methodology}a illustrates an $x$-oriented grid defined by $N_{x} = 4$ and $N_{y} = 3$.
In this example, the matrix coordinates $x_{7}$ and $y_{7}$, defined by the matrix index $i = 7$ (or $j = 7$), 
are equivalent to the grid coordinates $x_{3}$ and $y_{2}$, which are defined by the grid indices
$k = 3$ and $l = 2$, respectively. These indices are computed with equations \ref{eq:l-x-oriented}
and \ref{eq:k-x-oriented}, by using the matrix index $i = 7$ (or $j = 7$).

Now, consider the case in which the regular grid of data is oriented along 
the $y$-axis (Figure \ref{fig:methodology}b). For convenience, we call them $y$-\textit{oriented grids}.
Similarly to $x$-oriented grids, we have the following integer functions associated with
$y$-oriented grids:
\begin{equation}
i(k, l) = (k - 1) \, N_{y} + l \quad ,
\label{eq:i-y-oriented}
\end{equation}
\begin{equation}
k(i) = \Bigg\lceil \frac{i}{N_{y}} \Bigg\rceil
\label{eq:k-y-oriented}
\end{equation}
and
\begin{equation}
l(i) = i - \Bigg\lceil \frac{i}{N_{y}} \Bigg\rceil N_{y} + N_{y} \quad .
\label{eq:l-y-oriented}
\end{equation}
Figure \ref{fig:methodology}b illustrates an $y$-oriented grid defined by $N_{x} = 4$ and $N_{y} = 3$.
In this example, the matrix coordinates $x_{7}$ and $y_{7}$, defined by the matrix index 
$i = 7$ (or $j = 7$), are equivalent to the grid coordinates $x_{3}$ and $y_{1}$, which are 
defined by the grid indices $k = 3$ and $l = 1$, respectively. Differently from the example
shown in Figure \ref{fig:methodology}a, the grid indices of the present example are 
computed with equations \ref{eq:k-y-oriented} and \ref{eq:l-y-oriented}, by using the 
matrix index $i = 7$ (or $j = 7$).

The element $a_{ij}$ (equation \ref{eq:aij}) can be rewritten 
by using equations \ref{eq:xi} and \ref{eq:yi}, giving rise to:
\begin{equation}
a_{ij} = \frac{c_{g} \, G \, \Delta z}{ \left[ 
	\left( \Delta k_{ij} \, \Delta x \right)^{2} + 
	\left( \Delta l_{ij} \, \Delta y \right)^{2} + 
	\left( \Delta z \right)^{2} \right]^{\frac{3}{2}}} \: ,
\label{eq:aij-regular-grids}
\end{equation}
where $\Delta z = z_{0} - z_{1}$, 
$\Delta k_{ij} = k(i) - k(j)$ (equations \ref{eq:k-x-oriented} or \ref{eq:k-y-oriented}) and
$\Delta l_{ij} = l(i) - l(j)$ (equations \ref{eq:l-x-oriented} or \ref{eq:l-y-oriented}).
Notice that the structure of matrix $\mathbf{A}$ (equation \ref{eq:predicted-data-vector}) for 
the case in which its elements are given by $a_{ij}$ (equation \ref{eq:aij-regular-grids}) is 
defined by the coefficients $\Delta k_{ij}$ and $\Delta l_{ij}$.

For $x$-oriented grids, the coefficients $\Delta k_{ij}$ and $\Delta l_{ij}$ are 
computed by using equations \ref{eq:k-x-oriented} and \ref{eq:l-x-oriented}, respectively.
In this case, $\mathbf{A}$ (equation \ref{eq:predicted-data-vector}) is 
composed of $N_{y} \times N_{y}$ blocks, where each block is formed by $N_{x} \times N_{x}$ elements.
For $y$-oriented grids, the coefficients $\Delta k_{ij}$ and $\Delta l_{ij}$ are 
computed by using equations \ref{eq:k-y-oriented} and \ref{eq:l-y-oriented}, respectively.
In this case, $\mathbf{A}$ (equation \ref{eq:predicted-data-vector}) is a 
composed of $N_{x} \times N_{x}$ blocks, where each block is formed by $N_{y} \times N_{y}$ elements.
In both cases, $\mathbf{A}$ is Toeplitz blockwise, i.e., the blocks lying at the same block 
diagonal are equal to each other.
Besides, the blocks located above the main diagonal are equal to those 
located below and each block is itself a Toeplitz matrix.
These symmetries come from the fact that the coefficients
$\Delta k_{ij}$ and $\Delta l_{ij}$ are squared at the denominator of 
$a_{ij}$ (equation \ref{eq:aij-regular-grids}).
Matrices with this well-defined pattern are called 
Doubly Block Toeplitz \citep[][ p. 28]{jain1989} or symmetric Block-Toeplitz Toeplitz-Block (BTTB),
for example. We opted for using the second term.

This well-defined pattern is better represented by using the \textit{block indices} $q$ and $p$. 
We represent $\mathbf{A}$ (equation \ref{eq:predicted-data-vector}) as a grid of $Q \times Q$ blocks 
$\mathbf{A}_{q}$, $q = 0, \dots, Q - 1$, given by
\begin{equation}
	\mathbf{A} = \begin{bmatrix}
	\mathbf{A}_{0}   & \mathbf{A}_{1} & \cdots         & \mathbf{A}_{Q-1} \\
	\mathbf{A}_{1}   & \mathbf{A}_{0} & \ddots         & \vdots           \\ 
	\vdots           & \ddots         & \ddots         & \mathbf{A}_{1}   \\
	\mathbf{A}_{Q-1} & \cdots         & \mathbf{A}_{1} & \mathbf{A}_{0}                 
	\end{bmatrix}_{N \times N} \: ,
	\label{eq:BTTB_A}
\end{equation}
where each block has $P \times P$ elements conveniently represented by $a^{q}_{p}$, 
$p = 0, \dots, P - 1$, as follows:
\begin{equation}
	\mathbf{A}_{q} = \begin{bmatrix}
	a^{q}_{0}   & a^{q}_{1} & \cdots    & a^{q}_{P-1} \\
	a^{q}_{1}   & a^{q}_{0} & \ddots    & \vdots           \\ 
	\vdots      & \ddots    & \ddots    & a^{q}_{1}   \\
	a^{q}_{P-1} & \cdots    & a^{q}_{1} & a^{q}_{0}                 
	\end{bmatrix}_{P \times P} \: ,
	\label{eq:Aq_block}
\end{equation}
with $N = QP$. The index $q$ defines the block diagonal where $\mathbf{A}_{q}$ (equation \ref{eq:Aq_block}) 
lies within the BTTB matrix $\mathbf{A}$ (equation \ref{eq:BTTB_A}). 
This index varies from $0$, at the main diagonal, to $Q - 1$, at
the corners of $\mathbf{A}$. Similarly, the index $p$ defines the diagonal where $a^{q}_{p}$ 
lies within $\mathbf{A}_{q}$ (equation \ref{eq:Aq_block}). This index varies from $0$, 
at the main diagonal, to $P - 1$, at the corners of $\mathbf{A}_{q}$.
For $x$-oriented grids, $Q = N_{y}$, $P = N_{x}$ and the block indices
$q$ and $p$ are defined, respectively, by the following integer functions 
of the matrix indices $i$ and $j$:
\begin{equation}
q(i, j) = \; \mid l(i) - l(j) \mid
\label{eq:q-x-oriented}
\end{equation}
and
\begin{equation}
p(i, j) = \; \mid k(i) - k(j) \mid \quad ,
\label{eq:p-x-oriented}
\end{equation}
where $l(i)$ and $l(j)$ are defined by equation \ref{eq:l-x-oriented} 
and $k(i)$ and $k(j)$ are defined by equation \ref{eq:k-x-oriented}.
For $y$-oriented grids, $Q = N_{x}$, $P = N_{y}$ and the block indices
$q$ and $p$ are defined, respectively, by the following integer functions 
of the matrix indices $i$ and $j$:
\begin{equation}
q(i, j) = \; \mid k(i) - k(j) \mid 
\label{eq:q-y-oriented}
\end{equation}
and
\begin{equation}
p(i, j) = \; \mid l(i) - l(j) \mid \quad ,
\label{eq:p-y-oriented}
\end{equation}
where $k(i)$ and $k(j)$ are defined by equation \ref{eq:k-y-oriented}
and $l(i)$ and $l(j)$ are defined by equation \ref{eq:l-y-oriented}.
Notice that, for each element $a_{ij}$ (equation \ref{eq:aij-regular-grids}),
defined by matrix indices $i$ and $j$, there is a corresponding block element
$a^{q}_{p}$, defined by block indices $q$ (equations \ref{eq:q-x-oriented} or 
\ref{eq:q-y-oriented}) and $p$ (equations \ref{eq:p-x-oriented} or \ref{eq:p-y-oriented}),
so that
\begin{equation}
	a^{q}_{p} \equiv a_{ij} \quad .
	\label{eq:aqp_equiv_aij}
\end{equation}
We also stress that matrix $\mathbf{A}$ (equation \ref{eq:predicted-data-vector}) defined 
by elements $a_{ij}$ (equation \ref{eq:aij-regular-grids}) in terms of matrix indices 
$i$ and $j$ is strictly the same BTTB matrix $\mathbf{A}$ (equation \ref{eq:BTTB_A}) defined 
by the blocks $\mathbf{A}_{q}$ (equation \ref{eq:Aq_block}) and block elements 
$a^{q}_{p}$ (equation \ref{eq:aqp_equiv_aij}) in terms of the block indices 
$q$ (equations \ref{eq:q-x-oriented} or \ref{eq:q-y-oriented}) and $p$ 
(equations \ref{eq:p-x-oriented} or \ref{eq:p-y-oriented}).

It is important noting that different matrix indices $i$ or $j$ produce the same 
absolute values for the grid indices $k$ (equations \ref{eq:k-x-oriented} or 
\ref{eq:k-y-oriented}) and $l$ (equations \ref{eq:l-x-oriented} or 
\ref{eq:l-y-oriented}). As a consequence, different pairs of matrix indices $i$
and $j$ generate the same absolute values for the coefficients $\Delta k_{ij}$ and
$\Delta l_{ij}$ that compose the denominator of $a_{ij}$ 
(equation \ref{eq:aij-regular-grids}) and also the same values for the block indices
$q$ (equations \ref{eq:q-x-oriented} or \ref{eq:q-y-oriented}) and 
$p$ (equations \ref{eq:p-x-oriented} or \ref{eq:p-y-oriented}), as well as the same
block element $a^{q}_{p}$ (equation \ref{eq:aqp_equiv_aij}). 
It means that elements $a_{ij}$ defined
by different matrix indices $i$ and $j$ have the same value. The key point for
understanding the structure of BTTB matrix $\mathbf{A}$ (equation \ref{eq:BTTB_A})
is then, given a single element $a_{ij}$
defined by matrix indices $i$ and $j$, compute the grid indices 
$k$ (equations \ref{eq:k-x-oriented} or \ref{eq:k-y-oriented}) and
$l$ (equations \ref{eq:l-x-oriented} or \ref{eq:l-y-oriented}).
These grid indices are used to 
(1) compute the coefficients $\Delta k_{ij}$ and 
$\Delta l_{ij}$ and determine the value of $a_{ij}$ with equation 
\ref{eq:aij-regular-grids} and 
(2) compute the block indices $q$ (equations \ref{eq:q-x-oriented} 
or \ref{eq:q-y-oriented}) and $p$ (equations \ref{eq:p-x-oriented} or 
\ref{eq:p-y-oriented}) and determine the corresponding block element $a^{q}_{p}$ 
(equation \ref{eq:aqp_equiv_aij}) forming $\mathbf{A}_{q}$ (equation \ref{eq:Aq_block}). 

Consider the $x$-oriented grid of $N_{x} \times N_{y}$ points shown in Figure \ref{fig:methodology}a, 
with $N_{x} = 4$, $N_{y} = 3$ and $N = N_{x} \, N_{y} = 12$.
To illustrate the relationship between the matrix indices ($i$ and $j$) and 
the block indices ($q$ and $p$), consider the element $a_{ij}$ defined by 
$i = 2$ and $j = 10$, which is
located in the 2nd line and 10th column of $\mathbf{A}$ (equation \ref{eq:predicted-data-vector}).
By using equations \ref{eq:l-x-oriented} and \ref{eq:k-x-oriented}, we obtain the 
grid indices $l(i) = 1$, $l(j) = 3$, $k(i) = 2$ and $k(j) = 2$.
These grid indices result in the coefficients $\Delta k_{ij} = 0$ and $\Delta l_{ij} = -2$,
which are used to compute the element $a_{ij}$ (equation \ref{eq:aij-regular-grids}),
as well as in the block indices $q = 2$ (equation \ref{eq:q-x-oriented}) and 
$p = 0$ (equation \ref{eq:p-x-oriented}).
These block indices indicate that this element $a_{ij}$ appears in the main diagonal
of the blocks $\mathbf{A}_{2}$ (equation \ref{eq:Aq_block}), which are located at the corners 
of $\mathbf{A}$ (equation \ref{eq:BTTB_A}).
To verify this, let us take the matrix indices associated with these elements.
They are $(i, j)$ = $(1, 9)$, $(2, 10)$, $(3, 11)$, $(4, 12)$, $(9, 1)$, $(10, 2)$, 
$(11, 3)$ and $(12, 4)$. By using these matrix indices, it is easy to verify that all
of them produce the same grid indices $l(i)$, $l(j)$, $k(i)$ and $k(j)$ 
(equations \ref{eq:l-x-oriented} and \ref{eq:k-x-oriented}) as those associated with
the element defined by $i = 2$ and $j = 10$ or vice versa ($i = 10$ and $j = 2$). 
Consequently, all of them produce
elements $a_{ij}$ (equation \ref{eq:aij-regular-grids}) having the same value.
Besides, it is also easy to verify that all these matrix indices produce the same block
indices $q = 2$ (equation \ref{eq:q-x-oriented}) and $p = 0$ (equation \ref{eq:p-x-oriented})
and the same block element $a^{q}_{p}$ (equation \ref{eq:aqp_equiv_aij}).
By repeating this procedure for all elements $a_{ij}$, $i = 1, \dots, 12$, $j = 1, \dots, 12$, 
forming the matrix $\mathbf{A}$ (equation \ref{eq:predicted-data-vector}) obtained from our 
$x$-oriented grid (Figure \ref{fig:methodology}a), we can verify that
\begin{equation}
\mathbf{A} = \begin{bmatrix}
\mathbf{A}_{0} & \mathbf{A}_{1} & \mathbf{A}_{2} \\
\mathbf{A}_{1} & \mathbf{A}_{0} & \mathbf{A}_{1} \\
\mathbf{A}_{2} & \mathbf{A}_{1} & \mathbf{A}_{0}
\end{bmatrix}_{N \times N} \quad ,
\label{eq:A-x-oriented-example}
\end{equation}
where $\mathbf{A}_{q}$ (equation \ref{eq:Aq_block}), $q = 0, \dots, Q -1$, $Q = N_{y}$, 
are symmetric Toeplitz matrices given by:
\begin{equation}
\mathbf{A}_{q} = \begin{bmatrix}
a^{q}_{0} & a^{q}_{1} & a^{q}_{2} & a^{q}_{3} \\
a^{q}_{1} & a^{q}_{0} & a^{q}_{1} & a^{q}_{2} \\
a^{q}_{2} & a^{q}_{1} & a^{q}_{0} & a^{q}_{1} \\
a^{q}_{3} & a^{q}_{2} & a^{q}_{1} & a^{q}_{0}
\end{bmatrix}_{N_{x} \times N_{x}} \quad ,
\label{eq:Aq-x-oriented}
\end{equation}
with elements $a^{q}_{p}$ (equation \ref{eq:aqp_equiv_aij}) defined by 
$p = 0, \dots, P - 1$, $P = N_{x}$.

This procedure can also be used to verify that the matrix $\mathbf{A}$ 
(equation \ref{eq:predicted-data-vector}) obtained
from the $y$-oriented grid illustrated in Figure \ref{fig:methodology}b is given by
\begin{equation}
\mathbf{A} = \begin{bmatrix}
\mathbf{A}_{0} & \mathbf{A}_{1} & \mathbf{A}_{2} & \mathbf{A}_{3} \\
\mathbf{A}_{1} & \mathbf{A}_{0} & \mathbf{A}_{1} & \mathbf{A}_{2} \\
\mathbf{A}_{2} & \mathbf{A}_{1} & \mathbf{A}_{0} & \mathbf{A}_{1} \\
\mathbf{A}_{3} & \mathbf{A}_{2} & \mathbf{A}_{1} & \mathbf{A}_{0}
\end{bmatrix}_{N \times N} \quad ,
\label{eq:A-y-oriented-example}
\end{equation}
where $\mathbf{A}_{q}$ (equation \ref{eq:Aq_block}), $q = 0, \dots, Q - 1$, $Q = N_{x}$, 
are symmetric Toeplitz matrices given by:
\begin{equation}
\mathbf{A}_{q} = \begin{bmatrix}
a^{q}_{0} & a^{q}_{1} & a^{q}_{2} \\
a^{q}_{1} & a^{q}_{0} & a^{q}_{1} \\
a^{q}_{2} & a^{q}_{1} & a^{q}_{0}
\end{bmatrix}_{N_{y} \times N_{y}} \quad ,
\label{eq:Aq-y-oriented}
\end{equation}
with elements $a^{q}_{p}$ (equation \ref{eq:aqp_equiv_aij}) defined by 
$p = 0, \dots, P - 1$, $P = N_{y}$.

These examples (equations \ref{eq:A-x-oriented-example}--\ref{eq:Aq-y-oriented}) show 
that the entire $N \times N$ BTTB matrix $\mathbf{A}$ 
(equations \ref{eq:predicted-data-vector} and \ref{eq:BTTB_A}) 
can be defined by using only the elements 
forming its first column (or row). Notice that a column contains the gravitational effect 
produced by a single equivalent source at all $N$ observation points.

\subsection{BTTB matrix-vector product}

The matrix-vector product $\tensor{A} \hat{\mathbf{p}}^{k}$ (equation 
\ref{eq:delta_p_k_fast_eqlayer}) required by the fast equivalent-layer 
technique \citep{siqueira-etal2017} accounts for most of its total computation time 
and can cause RAM memory shortage when large data sets are used.
This computational load can be drastically reduced by exploring the well-defined structure of 
matrix $\mathbf{A}$ (equation \ref{eq:predicted-data-vector}) for the particular case in which 
its elements $a_{ij}$ are defined by equation \ref{eq:aij-regular-grids}. 
In this case, $\mathbf{A}$ is a symmetric BTTB matrix (equations \ref{eq:BTTB_A} and 
\ref{eq:A-x-oriented-example}--\ref{eq:Aq-y-oriented}) and the predicted data vector 
$\mathbf{d}(\mathbf{p})$ (equation \ref{eq:predicted-data-vector}) can be efficiently
computed by using the 2D Discrete Fourier Transform (DFT).
To do this, let us first rewrite $\mathbf{d}(\mathbf{p})$ and
$\mathbf{p}$ (equation \ref{eq:predicted-data-vector}) as the following partitioned vectors:
\begin{equation}
\mathbf{d}(\mathbf{p}) = \begin{bmatrix}
\mathbf{d}_{0}(\mathbf{p}) \\
\vdots \\
\mathbf{d}_{Q - 1}(\mathbf{p})
\end{bmatrix}_{N \times 1}
\label{eq:predicted-data-vector-partitioned}
\end{equation}
and
\begin{equation}
\mathbf{p} = \begin{bmatrix}
\mathbf{p}_{0} \\
\vdots \\
\mathbf{p}_{Q - 1}
\end{bmatrix}_{N \times 1} \quad ,
\label{eq:parameter-vector-partitioned}
\end{equation}
where $\mathbf{d}_{q}(\mathbf{p})$ and $\mathbf{p}_{q}$, $q = 0, \dots, Q - 1$,
are $P \times 1$ vectors. Notice that $q$ is the block index defined by equations 
\ref{eq:q-x-oriented} and \ref{eq:q-y-oriented}, $Q$ defines the number of blocks
$\mathbf{A}_{q}$ (equation \ref{eq:Aq_block}) forming $\mathbf{A}$ (equation \ref{eq:BTTB_A}) 
and $P$ defines the number of elements forming each block $\mathbf{A}_{q}$.
Then, by using the partitioned vectors 
(equations \ref{eq:parameter-vector-partitioned} and \ref{eq:predicted-data-vector-partitioned}) 
and remembering that $N = QP$, we define the auxiliary linear system
\begin{equation}
\mathbf{w} = \mathbf{C} \mathbf{v} \: ,
\label{eq:w_Cv}
\end{equation}
where
\begin{equation}
\mathbf{w} = \begin{bmatrix}
\mathbf{w}_{0} \\
\vdots \\
\mathbf{w}_{Q - 1} \\
\mathbf{0}_{2N \times 1}
\end{bmatrix}_{4N \times 1} \quad ,
\label{eq:w-vector}
\end{equation}
\begin{equation}
\mathbf{w}_{q} = \begin{bmatrix}
\mathbf{d}_{q}(\mathbf{p}) \\
\mathbf{0}_{P \times 1}
\end{bmatrix}_{2P \times 1}
\label{eq:wq-vector} \quad ,
\end{equation}
\begin{equation}
\mathbf{v} = \begin{bmatrix}
\mathbf{v}_{0} \\
\vdots \\
\mathbf{v}_{Q - 1} \\
\mathbf{0}_{2N \times 1}
\end{bmatrix}_{4N \times 1} \quad ,
\label{eq:v-vector}
\end{equation}
and
\begin{equation}
\mathbf{v}_{q} = \begin{bmatrix}
\mathbf{p}_{q} \\
\mathbf{0}_{P \times 1}
\end{bmatrix}_{2P \times 1}
\label{eq:vq-vector} \quad ,
\end{equation}
with $\mathbf{d}_{q}(\mathbf{p})$ and $\mathbf{p}_{q}$ defined by
equations \ref{eq:predicted-data-vector-partitioned} and 
\ref{eq:parameter-vector-partitioned}, respectively.
Finally $\mathbf{C}$ (equation \ref{eq:w_Cv}) is a 
$4N \times 4N$ symmetric Block Circulant matrix with Circulant Blocks (BCCB) 
\citep[][ p. 184]{davis1979}. A detailed description of this matrix and
some of its relevant properties are presented in Appendix A. 

What follows shows a step-by-step description of how we use the auxiliary 
system (equation \ref{eq:w_Cv}) to compute the matrix-vector product 
$\tensor{A} \hat{\mathbf{p}}^{k}$ (equation \ref{eq:delta_p_k_fast_eqlayer}) in 
a computationally efficient way by exploring the structure of matrix $\mathbf{C}$.
By substituting equation \ref{eq:C-diagonalized} in the auxiliary system 
(equation \ref{eq:w_Cv}) and premultiplying both
sides of the result by $\left(\mathbf{F}_{2Q} \otimes \mathbf{F}_{2P} \right)$
(see the details in Appendix A), we obtain
\begin{equation}
\boldsymbol{\Lambda} \left(\mathbf{F}_{2Q} \otimes \mathbf{F}_{2P} \right) 
\mathbf{v} = \left(\mathbf{F}_{2Q} \otimes \mathbf{F}_{2P} \right) 
\mathbf{w} \: .
\label{eq:vec-DFT-system}
\end{equation}
Now, by applying the $vec$-operator to both sides of equation \ref{eq:vec-DFT-system} 
(see the details in Appendix B), we obtain:
\begin{equation}
\mathbf{F}_{2Q}^{\ast} \left[ 
\mathbf{L} \circ \left(\mathbf{F}_{2Q} \, \mathbf{V} \, \mathbf{F}_{2P} \right) 
\right] \mathbf{F}_{2P}^{\ast} = \mathbf{W} \: ,
\label{eq:DFT-system}
\end{equation}
where ``$\circ$'' denotes the Hadamard product \citep[][ p. 298]{horn_johnson1991} and 
$\mathbf{L}$, $\mathbf{V}$ and $\mathbf{W}$ are $2Q \times 2P$ matrices obtained 
by rearranging, along their rows, the elements forming the diagonal of matrix 
$\boldsymbol{\Lambda}$, vector $\mathbf{v}$ and vector $\mathbf{w}$, respectively.
The left side of equation \ref{eq:DFT-system} contains the 2D 
Inverse Discrete Fourier Transform (IDFT) of the term in brackets, which in turn
represents the Hadamard product of matrix $\mathbf{L}$ (equation \ref{eq:left_side_DFT_system_3})
and the 2D DFT of matrix $\mathbf{V}$ (equation \ref{eq:left_side_DFT_system_1}).
Matrix $\mathbf{L}$ contains the eigenvalues 
of $\boldsymbol{\Lambda}$ (equation \ref{eq:C-diagonalized}) and can be 
efficiently computed by using only the first column of the BCCB matrix 
$\mathbf{C}$ (equation \ref{eq:w_Cv}) (see the details in Appendix C).
Here, we evaluate equation \ref{eq:DFT-system} and compute matrix $\mathbf{L}$
by using the 2D Fast Fourier Transform (2D FFT).
This approach, that have been used in potential-field methods
\citep[e.g.,][]{zhang-wong2015, zhang-etal2016, qiang_etal2019}, is actually 
a fast 2D discrete convolution \citep[e.g.,][ p. 213]{vanloan1992}.

At each iteration $k$th of the fast equivalent-layer technique, 
(equation \ref{eq:delta_p_k_fast_eqlayer}), we efficiently compute 
$\mathbf{A} \hat{\mathbf{p}}^{k} = \mathbf{d}(\hat{\mathbf{p}}^{k})$ by following 
the steps below:

\begin{itemize}
\item[\textbf{(1)}] Use equation \ref{eq:aij-regular-grids} to compute the first column 
of each block $\mathbf{A}_{q}$ (equation \ref{eq:Aq_block}), $q = 0, \dots, Q-1$, forming 
the BTTB matrix $\mathbf{A}$ (equation \ref{eq:BTTB_A});

\item[\textbf{(2)}] Rearrange the first column of $\mathbf{A}$ according to equations 
\ref{eq:C-first-column-blocks} and \ref{eq:Cq-first-column} to obtain the
first column $\mathbf{c}_{0}$ of the BCCB matrix $\mathbf{C}$ (equation \ref{eq:w_Cv});

\item[\textbf{(3)}] Rearrange $\mathbf{c}_{0}$ along the rows of matrix $\mathbf{G}$ and
use the 2D FFT to compute matrix $\mathbf{L}$ (equation \ref{eq:DTF_G});

\item[\textbf{(4)}] Rearrange the parameter vector $\hat{\mathbf{p}}^{k}$ 
(equation \ref{eq:predicted-data-vector}) in its partitioned form 
(equation \ref{eq:parameter-vector-partitioned}) to define the auxiliary vector 
$\mathbf{v}$ (equation \ref{eq:v-vector});

\item[\textbf{(5)}] Rearrange $\mathbf{v}$ to obtain matrix $\mathbf{V}$, use the 2D FFT 
to compute its DFT and evaluate the left side of equation \ref{eq:DFT-system-preliminary};

\item[\textbf{(6)}] Use the 2D FFT to compute the IDFT of the result obtained in step (5) to 
obtain the matrix $\mathbf{W}$ (equation \ref{eq:DFT-system});

\item[\textbf{(7)}] Use the $vec$-operator (equation \ref{eq:vec-operator}) and equations 
\ref{eq:w-vector} and \ref{eq:wq-vector} to rearrange $\mathbf{W}$ in order to obtain the predicted 
data vector $\mathbf{d}(\hat{\mathbf{p}}^{k})$.

\end{itemize}


\subsection{Computational performance}


The number of flops (floating-point operations) necessary to estimate the 
$N \times 1$ parameter vector $\mathbf{p}$ in the fast equivalent-layer technique 
\citep{siqueira-etal2017} is
\begin{equation}
f_{0} = N^{it} (3N + 2N^{2}) \; ,
\label{eq:float_fast_eqlayer}
\end{equation}
where $N^{it}$ is the number of iterations. In this equation, the term $2N^2$ is associated 
with the matrix-vector product $\mathbf{A} \hat{\mathbf{p}}^{k}$ (equation 
\ref{eq:delta_p_k_fast_eqlayer}) and accounts for most of the computational complexity 
of this method.
Our method replace this matrix-vector product by three operations: 
one DFT, one Hadamard product and one IDFT involving $2Q \times 2P$ matrices 
(left side of equation \ref{eq:DFT-system}). 
The Hadamard product requires $24N$ flops, $N = QP$, because the entries are 
complex numbers.
We consider that a DFT/IDFT requires $\kappa \, 4N \log_{2}(4N)$ flops to be computed via 2D FFT, 
where $\kappa$ is a constant depending on the algorithm. 
Then, the resultant flops count of our method is given by:
\begin{equation}
f_{1} = N^{it} \left[ 27N + \kappa \, 8N \log_{2}(4N) \right] \: .
\label{eq:float_bccb}
\end{equation}
Figure \ref{fig:float} shows the flops counts $f_{0}$ and $f_{1}$ (equations \ref{eq:float_fast_eqlayer}
and \ref{eq:float_bccb}) associated with the fast equivalent-layer technique \citep{siqueira-etal2017} and 
our method, respectively, as a function of the number $N$ of observation points. 
We considered a fixed number of $N^{it} = 50$ iterations and $\kappa = 5$ (equation \ref{eq:float_bccb}),
which is compatible with a radix-2 FFT \citep[][ p. 16]{vanloan1992}.
As we can see, the number of flops is drastically decreased in our method.

Another advantage of our method is concerned with the real $N \times N$ matrix $\mathbf{A}$ 
(equations \ref{eq:predicted-data-vector} and \ref{eq:BTTB_A}).
In the fast equivalent-layer technique, the full matrix 
is computed once and stored during the entire iterative process.
On the other hand, our method computes only one column of $\mathbf{A}$ 
and uses it to compute the complex $2Q \times 2P$ matrix $\mathbf{L}$ 
(equation \ref{eq:DTF_G}) via 2D FFT, which is stored during all iterations.
Table \ref{tab:RAM-usage} shows the RAM memory usage needed to store the 
full matrix $\mathbf{A}$, a single column of $\mathbf{A}$ and the full matrix 
$\mathbf{L}$. These quantities were computed for different numbers of observations $N$. 
Notice that $N = 1\,000\,000$ observations require nearly $7.6$ Terabytes of RAM memory 
to store the whole matrix $\mathbf{A}$.

Figure \ref{fig:time_fast_eqlayer_bccb} compares the running time of the fast equivalent-layer technique 
\citep{siqueira-etal2017} and of our method, considering a constant number of iterations $N^{it} = 50$. 
We used a PC with an Intel Core i7 4790@3.6GHz processor and 16 GB of RAM memory.
The computational efficiency of our approach is significantly superior to that of the 
fast equivalent-layer technique for a number of observations $N$ greater than $10\,000$. 
We could not perform this comparison with a number of observations greater than $22\,500$
due to limitations of our PC in storing the full matrix $\mathbf{A}$.
Figure \ref{fig:time_bccb} shows the running time of our method with a number of observations 
up to  $25$ millions. 
These results shows that, while the running time of our method is $\approx 30.9$ seconds for 
$N = 1\,000\,000$, the fast equivalent-layer technique spends $\approx 46.8$ seconds for $N = 22\,500$.